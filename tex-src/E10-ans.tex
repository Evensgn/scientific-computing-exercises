\documentclass[12pt, a4paper]{article}
\usepackage{ctex}

\usepackage{float}
\usepackage[super, square]{natbib}
\usepackage[margin = 1in]{geometry}
\usepackage{
  color,
  clrscode,
  amssymb,
  ntheorem,
  amsmath,
  listings,
  fontspec,
  xcolor,
  supertabular,
  multirow
}
\definecolor{bgGray}{RGB}{36, 36, 36}
\usepackage[
  colorlinks,
  linkcolor=bgGray,
  anchorcolor=blue,
  citecolor=black
]{hyperref}
\newfontfamily\courier{Courier}

\theoremstyle{margin}
\theorembodyfont{\normalfont}

\newtheorem{theorem}{Theorem}
\newtheorem{definition}[theorem]{Definition}
\newtheorem{example}[theorem]{Example}

\newcommand{\st}{\text{s.t.}}
\newcommand{\mn}{\mathnormal}
\newcommand{\tbf}{\textbf}
\newcommand{\fl}{\mathnormal{fl}}
\newcommand{\f}{\mathnormal{f}}
\newcommand{\g}{\mathnormal{g}}
\newcommand{\R}{\mathbf{R}}
\newcommand{\Q}{\mathbf{Q}}
\newcommand{\JD}{\textbf{D}}
\newcommand{\rd}{\mathrm{d}}
\newcommand{\str}{^*}
\newcommand{\vep}{\varepsilon}
\newcommand{\lhs}{\text{L.H.S}}
\newcommand{\rhs}{\text{R.H.S}}
\newcommand{\con}{\text{Const}}
\newcommand{\oneton}{1,\,2,\,\dots,\,n}
\newcommand{\aoneton}{a_1a_2\dots a_n}
\newcommand{\xoneton}{x_1,\,x_2,\,\dots,\,x_n}


\title{科学计算 Exercise 10}
\author{范舟\\516030910574\\致远学院2016级ACM班}
\date{}

\begin{document}
\lstset{numbers=left,
  basicstyle=\scriptsize\courier,
  numberstyle=\tiny\courier\color{red!89!green!36!blue!36},
  language=Matlab,
  breaklines=true,
  keywordstyle=\color{blue!70},commentstyle=\color{red!50!green!50!blue!50},
  morekeywords={},
  stringstyle=\color{purple},
  frame=shadowbox,
  rulesepcolor=\color{red!20!green!20!blue!20}
}
\maketitle

\paragraph{3.} \textbf{证明} \\
设线性方程组的系数矩阵为$\mathbf{A}$,则
\[\mathbf{A} = \begin{bmatrix}
a_{11} & 0\\ 
0 & a_{22}
\end{bmatrix} - \begin{bmatrix}
0 & 0\\ 
-a_{21} & 0
\end{bmatrix} - \begin{bmatrix}
0 & -a_{12}\\ 
0 & 0
\end{bmatrix} \equiv \mathbf{D} - \mathbf{L} - \mathbf{U}\]
设迭代方程为
\[\mathbf{x}^{(k + 1)} = \mathbf{B}\mathbf{x}^{(k)} + \mathbf{f}, \quad k = 0, 1, \dots\]
若使用雅可比迭代法,则
\[\mathbf{B_1} = \mathbf{D}^{-1}(\mathbf{L} + \mathbf{U}) = \begin{bmatrix}
0 & -\frac{a_{12}}{a_{11}}\\ 
-\frac{a_{21}}{a_{22}} & 0
\end{bmatrix}\]
求得其特征值为
\[\bigg|\frac{a_{12}a_{21}}{a_{11}a_{22}}\bigg|^{\frac{1}{2}}, \quad -\bigg|\frac{a_{12}a_{21}}{a_{11}a_{22}}\bigg|^{\frac{1}{2}}\]
则有
\[\rho(\mathbf{B_1}) = \bigg|\frac{a_{12}a_{21}}{a_{11}a_{22}}\bigg|^{\frac{1}{2}}\]
若使用高斯-塞德尔迭代法,则
\[\mathbf{B_2} = (\mathbf{D} - \mathbf{L})^{-1}\mathbf{U} = \begin{bmatrix}
0 & -\frac{a_{12}}{a_{11}}\\ 
0 & -\frac{a_{12}a_{21}}{a_{11}a_{22}}
\end{bmatrix}\]
求得其特征值为
\[0, \quad -\frac{a_{12}a_{21}}{a_{11}a_{22}}\]
则有
\[\rho(\mathbf{B_2}) = \bigg|\frac{a_{12}a_{21}}{a_{11}a_{22}}\bigg|\]
因此
\[\rho(\mathbf{B_1}) < 1 \quad \Leftrightarrow \quad \rho(\mathbf{B_2}) < 1\]
即证得此方程组的雅可比迭代法与高斯-塞德尔迭代法同时收敛或发散. \\
若两种迭代法收敛(即$\rho(\mathbf{B_1}) < 1$),则二者的渐进收敛速度之比为
\[\frac{R(\mathbf{B_1})}{R(\mathbf{B_2})} = \frac{\ln \rho(\mathbf{B_1})}{\ln \rho(\mathbf{B_2})} = \frac{1}{2}\]
\newline

\paragraph{4.} \textbf{解} \\
\[\mathbf{A} = \begin{bmatrix}
10 & 0 & 0\\
0 & 10 & 0\\ 
0 & 0 & 5
\end{bmatrix} - \begin{bmatrix}
0 & 0 & 0\\
-b & 0 & 0\\ 
0 & -a & 0
\end{bmatrix} - \begin{bmatrix}
0 & -a & 0\\
0 & 0 & -b\\ 
0 & 0 & 0
\end{bmatrix} \equiv \mathbf{D} - \mathbf{L} - \mathbf{U}\]
若使用雅可比迭代法,则
\[\mathbf{B_1} = \mathbf{D}^{-1}(\mathbf{L} + \mathbf{U}) = \begin{bmatrix}
0 & -\frac{a}{10} & 0\\
-\frac{b}{10} & 0 & -\frac{b}{10}\\ 
0 & -\frac{a}{5} & 0
\end{bmatrix}\]
求得其特征值为
\[0, \quad -\frac{(3ab)^{\frac{1}{2}}}{10}, \quad \frac{(3ab)^{\frac{1}{2}}}{10}\]
则有
\[\rho(\mathbf{B_1}) = \bigg|\frac{(3ab)^{\frac{1}{2}}}{10}\bigg|\]
因此雅可比迭代法收敛的充分必要条件为
\[|ab| < \frac{100}{3}\]
若使用高斯-塞德尔迭代法,则
\[\mathbf{B_2} = \mathbf{D}^{-1}(\mathbf{L} + \mathbf{U}) = \begin{bmatrix}
0 & -\frac{a}{10} & 0\\
0 & \frac{ab}{100} & -\frac{b}{10}\\ 
0 & -\frac{a^2b}{500} & \frac{ab}{50}
\end{bmatrix}\]
求得其特征值为
\[0, \quad 0, \quad \frac{3ab}{100}\]
则有
\[\rho(\mathbf{B_2}) = \bigg|\frac{3ab}{100}\bigg|\]
因此高斯-塞德尔迭代法收敛的充分必要条件为
\[|ab| < \frac{100}{3}\]
\newline

\paragraph{6.} \textbf{解} \\
\[\mathbf{A} = \begin{bmatrix}
3 & 0 & 0\\
0 & 2 & 0\\ 
0 & 0 & 2
\end{bmatrix} - \begin{bmatrix}
0 & 0 & 0\\
0 & 0 & 0\\ 
2 & -1 & 0
\end{bmatrix} - \begin{bmatrix}
0 & 0 & 2\\
0 & 0 & -1\\ 
0 & 0 & 0
\end{bmatrix} \equiv \mathbf{D} - \mathbf{L} - \mathbf{U}\]
若使用雅可比迭代法,则
\[\mathbf{B_1} = \mathbf{D}^{-1}(\mathbf{L} + \mathbf{U}) = \begin{bmatrix}
0 & 0 & \frac{2}{3}\\
0 & 0 & -\frac{1}{2}\\ 
1 & -\frac{1}{2} & 0
\end{bmatrix}\]
求得其特征值为
\[0, \quad -0.9574, \quad 0.9574\]
则有
\[\rho(\mathbf{B_1}) = 0.9574 < 1\]
因此雅可比迭代法收敛.
若使用高斯-塞德尔迭代法,则
\[\mathbf{B_2} = \mathbf{D}^{-1}(\mathbf{L} + \mathbf{U}) = \begin{bmatrix}
0 & 0 & \frac{2}{3} \\
0 & 0 & -\frac{1}{2} \\ 
0 & 0 & \frac{11}{12}
\end{bmatrix}\]
求得其特征值为
\[0, \quad 0, \quad 0.9167\]
则有
\[\rho(\mathbf{B_2}) = 0.9167 < 1\]
因此高斯-塞德尔迭代法收敛. \\
又因为$\rho(\mathbf{B_2}) < \rho(\mathbf{B_1})$,因此高斯-塞德尔迭代法的收敛速度快.
\newline

\paragraph{9.} \textbf{证明} \\
将迭代公式改写为标准形式
\[\mathbf{x}^{(k + 1)} = (\mathbf{I} - \omega\mathbf{A}) \mathbf{x}^{(  k)} + \omega \mathbf{b}\]
令$\mathbf{B} = \mathbf{I} - \omega\mathbf{A}$,则有
\[\lambda(\mathbf{B}) = 1 - \omega \lambda(\mathbf{A})\]
若迭代法收敛,有$|\lambda(\mathbf{B})| < 1$,则
\[-1 < 1 - \omega \lambda(\mathbf{A}) < 1 \quad \Rightarrow \quad 0 < \omega\lambda(\mathbf{A}) < 2\]
因为$\mathbf{A}$为正定矩阵,有$\lambda(\mathbf{A}) > 0$,则有
\[0 < \omega < \frac{2}{\lambda(\mathbf{A})}\]
因为$\max \lambda(\mathbf{A}) = \beta$,只需$0 < \omega < \frac{2}{\beta}$,迭代法即收敛,证毕.

\end{document}