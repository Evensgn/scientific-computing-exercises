\documentclass[12pt, a4paper]{article}
\usepackage{ctex}

\usepackage{float}
\usepackage[super, square]{natbib}
\usepackage[margin = 1in]{geometry}
\usepackage{
  color,
  clrscode,
  amssymb,
  ntheorem,
  amsmath,
  listings,
  fontspec,
  xcolor,
  supertabular,
  multirow
}
\definecolor{bgGray}{RGB}{36, 36, 36}
\usepackage[
  colorlinks,
  linkcolor=bgGray,
  anchorcolor=blue,
  citecolor=black
]{hyperref}
\newfontfamily\courier{Courier}

\theoremstyle{margin}
\theorembodyfont{\norm  alfont}

\newtheorem{theorem}{Theorem}
\newtheorem{definition}[theorem]{Definition}
\newtheorem{example}[theorem]{Example}

\newcommand{\st}{\text{s.t.}}
\newcommand{\mn}{\mathnormal}
\newcommand{\tbf}{\textbf}
\newcommand{\fl}{\mathnormal{fl}}
\newcommand{\f}{\mathnormal{f}}
\newcommand{\g}{\mathnormal{g}}
\newcommand{\R}{\mathbf{R}}
\newcommand{\Q}{\mathbf{Q}}
\newcommand{\JD}{\textbf{D}}
\newcommand{\rd}{\mathrm{d}}
\newcommand{\str}{^*}
\newcommand{\vep}{\varepsilon}
\newcommand{\lhs}{\text{L.H.S}}
\newcommand{\rhs}{\text{R.H.S}}
\newcommand{\con}{\text{Const}}
\newcommand{\oneton}{1,\,2,\,\dots,\,n}
\newcommand{\aoneton}{a_1a_2\dots a_n}
\newcommand{\xoneton}{x_1,\,x_2,\,\dots,\,x_n}


\title{科学计算 Exercise 06c}
\author{范舟\\516030910574\\致远学院2016级ACM班}
\date{}

\begin{document}

\lstset{numbers=left,
  basicstyle=\scriptsize\courier,
  numberstyle=\tiny\courier\color{red!89!green!36!blue!36},
  language=C++,
  breaklines=true,
  keywordstyle=\color{blue!70},commentstyle=\color{red!50!green!50!blue!50},
  morekeywords={},
  stringstyle=\color{purple},
  frame=shadowbox,
  rulesepcolor=\color{red!20!green!20!blue!20}
}
\maketitle

1. \paragraph{解:} 高斯-勒让德公式为
\[\int_{-1}^{1} f(x) \rd x \approx \sum_{k = 0}^{n} A_k f(x_k)\]
将原积分转化为$[-1, 1]$区间上的积分
\[\int_{1}^{3} e^x \sin{x} \rd x = \int_{-1}^{1} e^{x + 2} \sin{(x + 2)} \rd x\]
$n = 2$时,有
\[\begin{split}
& x_0 = -0.7745967, x_1 = 0, x_2 = 0.7745967 \\
& A_0 = 0.5555556, A_1 = 0.8888889, A_2 = 0.5555556
\end{split}\]
函数在各高斯点的取值为
\[\begin{cases}
f(x_0) = 3.204416487 \\
f(x_1) = 6.718849697 \\
f(x_2) = 5.752548197
\end{cases}\]
代入高斯-勒让德公式,得
\[I \approx \sum_{k = 0}^{2} A_k f(x_k) = 10.94840\]
$n = 3$时,有
\[\begin{split}
& x_0 = -0.8611363, x_1 = -0.3399810, x_2 = 0.3399810, x_3 = 0.8611363 \\
& A_0 = 0.3478548, A_1 = 0.6521452, A_2 = 0.6521452, A_3 = 0.3478548
\end{split}\]
函数在各高斯点的取值为
\[\begin{cases}
f(x_0) = 2.836376140 \\
f(x_1) = 5.238490398 \\
f(x_2) = 7.458548459 \\
f(x_3) = 4.838744537
\end{cases}\]
代入高斯-勒让德公式,得
\[I \approx \sum_{k = 0}^{3} A_k f(x_k) = 10.95014\]
\newline 

2. \paragraph{解:} 由题目条件计算得
\[a = 7782.5, \quad b = 972.5\]
由于给定的数据精度为$1 \mathrm{km}$,只需在此精度下计算数值积分
\[S = 4a \int_{0}^{\pi / 2} \sqrt{1 - (\frac{c}{a})^2 \sin^2{\theta}} \rd \theta = 31130 \int_{0}^{\pi / 2} \sqrt{1 - (\frac{389}{3113})^2 \sin^2{\theta}} \rd \theta\]
使用$n = 4$的柯特斯公式的余项为
\[R[f] = -\frac{2(b - a)}{945} \Big(\frac{b - a}{4}\Big)^6 f^{(6)}(\eta) < 0.5, \quad \eta \in (0, \pi / 2)\]
由$n = 4$的柯特斯公式计算得
\[\begin{split}
C & = \frac{b - a}{90}[7 f(x_0) + 32 f(x_1) + 12 f(x_2) + 32 f(x_3) + 7 f(x_4)] \\
& = \frac{\pi}{180}[7 * 31130 + 32 * 31094.39 + 12 * 31008.24 + 32 * 30921.85 + 7 * 30886] \\
& = 48707.4
\end{split}\]
即卫星轨道周长约为$48707 \mathrm{km}$.
\end{document}