\documentclass[12pt, a4paper]{article}
\usepackage{ctex}

\usepackage{float}
\usepackage[super, square]{natbib}
\usepackage[margin = 1in]{geometry}
\usepackage{
  color,
  clrscode,
  amssymb,
  ntheorem,
  amsmath,
  listings,
  fontspec,
  xcolor,
  supertabular,
  multirow
}
\definecolor{bgGray}{RGB}{36, 36, 36}
\usepackage[
  colorlinks,
  linkcolor=bgGray,
  anchorcolor=blue,
  citecolor=black
]{hyperref}
\newfontfamily\courier{Courier}

\theoremstyle{margin}
\theorembodyfont{\normalfont}

\newtheorem{theorem}{Theorem}
\newtheorem{definition}[theorem]{Definition}
\newtheorem{example}[theorem]{Example}

\newcommand{\st}{\text{s.t.}}
\newcommand{\mn}{\mathnormal}
\newcommand{\tbf}{\textbf}
\newcommand{\fl}{\mathnormal{fl}}
\newcommand{\f}{\mathnormal{f}}
\newcommand{\g}{\mathnormal{g}}
\newcommand{\R}{\mathbf{R}}
\newcommand{\Q}{\mathbf{Q}}
\newcommand{\JD}{\textbf{D}}
\newcommand{\rd}{\mathrm{d}}
\newcommand{\str}{^*}
\newcommand{\vep}{\varepsilon}
\newcommand{\lhs}{\text{L.H.S}}
\newcommand{\rhs}{\text{R.H.S}}
\newcommand{\con}{\text{Const}}
\newcommand{\oneton}{1,\,2,\,\dots,\,n}
\newcommand{\aoneton}{a_1a_2\dots a_n}
\newcommand{\xoneton}{x_1,\,x_2,\,\dots,\,x_n}


\title{科学计算 Exercise 5}
\author{范舟\\516030910574\\致远学院2016级ACM班}
\date{}

\begin{document}

\lstset{numbers=left,
  basicstyle=\scriptsize\courier,
  numberstyle=\tiny\courier\color{red!89!green!36!blue!36},
  language=C++,
  breaklines=true,
  keywordstyle=\color{blue!70},commentstyle=\color{red!50!green!50!blue!50},
  morekeywords={},
  stringstyle=\color{purple},
  frame=shadowbox,
  rulesepcolor=\color{red!20!green!20!blue!20}
}
\maketitle

1. \textbf{证明: } 假设在$[a, b]$上$f\not\equiv0$,则$\exists x_0 \in [a, b], f(x_0) \neq 0$,不妨设$f(x_0) < 0$,因为$f \in C[a, b]$,存在$\delta > 0$,使得
\[f(x) < 0, \quad \forall x \in (x_0 - \delta, x_0 + \delta)\]
取$g$为
\[g(x)=
\begin{cases}
[x - (x_0 - \delta)]^3 [x - (x_0 + \delta)]^3, \quad & x \in (x_0 - \delta, x_0 + \delta) \\
0, \quad & x \in [a, x_0 - \delta] \cup [x_0 + \delta, b]
\end{cases}
\]
则有$g \in C^{2}_{0}[a, b]$,且
\[\quad g(x) < 0, \quad \forall x \in (x_0 - \delta, x_0 + \delta)\]
因此
\[\int_{a}^{b}f(x)g(x)\mathrm{d}x = \int_{x_0 - \delta}^{x_0 + \delta}f(x)g(x)\mathrm{d}x > 0\]
与$\int_{a}^{b}f(x)g(x)\mathrm{d}x = 0$矛盾,因此在$[a, b]$上$f \equiv 0$.
\newline

2. \textbf{证明: } 假设此变分问题有解$y^{*} = f^{*}(x)$,根据守恒律定理,沿着曲线$y^{*} = f^{*}(x)$,成立
\[H = f'\frac{\partial L}{\partial f'} - L = Constant\]
其中
\[L(x, f, f') = x^2\big(f'(x)\big)^2\]
则有
\[H = f'(x) \times 2x^2f'(x) - x^2\big(f'(x)\big)^2 = x^2\big(f'(x)\big)^2 = Constant\]
代入$x = 0$得$H \equiv 0$,则有
\[f'(x) = 0, \quad \forall x \in [-1, 1], x \ne 0\]
又因为$f \in C^{1}[-1, 1]$,$f'(0) = 0$,即
\[f'(x) = 0, \quad \forall x \in [-1, 1]\]
则$f(x)$是$[-1, 1]$上的常数函数,与$f(-1) = -1, f(1) = 1$矛盾,因此此变分问题无解.
\newline

3. \textbf{解: }
\[y(1 + (\frac{\mathrm{d}y}{\mathrm{d}x})^2) = c \Rightarrow \frac{\mathrm{d}y}{\mathrm{d}x} = \sqrt{\frac{c}{y} - 1} \Rightarrow \mathrm{d}x = \sqrt{\frac{y}{c - y}} \mathrm{d}y\]
已知$y(0) = 0$,两边积分得
\[\int_{0}^{x} \mathrm{d}t = \int_{y(0)}^{y(x)} \sqrt{\frac{t}{c - t}} \mathrm{d}t \Rightarrow x = \int_{0}^{y}\sqrt{\frac{t}{c-t}} \mathrm{d}t\]
令$p = \sqrt{\frac{t}{c - t}}$,则$t = \frac{cp^2}{1 + p^2}$,得
\[x = \int_{0}^{\sqrt{\frac{y}{c - y}}} p \mathrm{d}\Big(\frac{cp^2}{1 + p^2}\Big) = \int_{0}^{\sqrt{\frac{y}{c - y}}} \frac{2cp^2}{(1 + p^2)^2} \mathrm{d}p = 2c \int_{0}^{\sqrt{\frac{y}{c - y}}} \frac{p^2}{(1 + p^2)^2} \mathrm{d}p\]
令$p = \tan \theta$,得
\[x = 2c \int_{0}^{\arctan \sqrt{\frac{y}{c - y}}} \frac{\tan^2 \theta}{\sec^4 \theta} \sec^2 \theta \mathrm{d}\theta = 2c \int_{0}^{\arctan \sqrt{\frac{y}{c - y}}} \sin^2 \theta \mathrm{d}\theta\]
令$\theta = \frac{\alpha}{2}$,得
\[x = \frac{c}{2} \int_{0}^{2 \arctan{\sqrt{\frac{y}{c - y}}}} (1 - \cos \alpha) \mathrm{d}\alpha = c \arctan{\sqrt{\frac{y}{c - y}}} - \frac{c}{2}\sin{(2\arctan{\sqrt{\frac{y}{c - y}}})}\]
已知$y(x_1) = y_1$,有
\[x_1 = c \arctan{\sqrt{\frac{y_1}{c - y_1}}} - \frac{c}{2}\sin{(2\arctan{\sqrt{\frac{y_1}{c - y_1}}})}\]
由此可确定常数$c$.
\newline

4. \textbf{解: } 不妨假设$f(0) = \tilde y_0, f(1) = \tilde y_n$,分别定义数据保真项与正则化项
\[J_1(f) = \sum_{i = 1}^{n - 1} \frac{h_{i} + h_{i + 1}}{2}[\tilde y_i - f(x_i)]^2, \quad J_2(f) = \int_{0}^{1} \big(f'(x)\big)^2 \mathrm{d}x\]
其中
\[h_i = x_i - x_{i - 1}, \quad 1 \le i \le n\]
所求最优化函数为
\[J(f) = J_1(f) + \alpha J_2(f), \quad f_* = \arg \min_{f} J(f)\]
$\forall \eta \in K_0, \varepsilon \in R$,$J(f)$在$f_*$处沿$\eta$的方向导数为
\[\frac{\mathrm{d}}{\mathrm{d}\varepsilon}J(f_* + \varepsilon\eta) = -2 \sum_{i = 1}^{n - 1} \frac{h_{i} + h_{i + 1}}{2}[\tilde y_i - f_*(x_i) - \varepsilon\eta(x_i)]\eta(x_i) + 2\alpha \int_{0}^{1} [f_*'(x) + \varepsilon\eta'(x)]\eta'(x) \mathrm{d}x\]
在$\varepsilon = 0$处有$\frac{\mathrm{d}}{\mathrm{d}\varepsilon}J(f^* + \varepsilon\eta) = 0$,即
\[-2 \sum_{i = 1}^{n - 1} \frac{h_{i} + h_{i + 1}}{2}[\tilde y_i - f_*(x_i)]\eta(x_i) + 2\alpha \int_{0}^{1} f_*'(x)\eta'(x) \mathrm{d}x = 0\]
下面通过选取合适的$\eta$导出$f_*$满足的条件.
\paragraph{Step 1} 对于每个区间$[x_{k - 1}, x_k]$,取$\eta \in C_{0}^{\infty}(x_{k - 1}, x_k)$(定义在$[0, 1]$上,在$(x_{k - 1}, x_k)$的各阶导数都有紧支集),则有
\[\int_{x_{k - 1}}^{x_{k}} f_*'(x)\eta'(x) \mathrm{d}x = 0\]
分部积分得
\[\int_{x_{k - 1}}^{x_{k}} f_*'(x)\eta'(x) \mathrm{d}x = f_*'(x)\eta(x) \bigg|_{x_{k-1}}^{x_{k}} - \int_{x_{k - 1}}^{x_{k}} f_*''(x)\eta(x) \mathrm{d}x = \int_{x_{k - 1}}^{x_{k}} f_*''(x)\eta(x) \mathrm{d}x = 0\]
$\eta$在$K_0[x_{k - 1}, x_{k}]$中有任意性,根据变分引理,得
\[f_*''(x) \equiv 0 \Rightarrow f_* \in P_1, \quad x \in [x_{k - 1}, x_{k}]\]
因此$f_*$是分段线性函数. 
\paragraph{Step 2} 再取$g \in C^{\infty}[0, 1] \cap K_0$,分部积分有
\[\begin{split}
& \int_{0}^{1} f_*'(x)\eta'(x) \rd x \\
& = \sum_{i = 1}^{n} \int_{x_{i - 1}}^{x_{i}} f_*'(x)\eta'(x) \rd x = \sum_{i = 1}^{n} \bigg(f_*'(x)\eta(x) \bigg|_{x_{i - 1}}^{x_i} - \int_{x_{i - 1}}^{x_{i}} f_*''(x)\eta(x) \rd x\bigg) \\
& = \sum_{i = 1}^{n - 1} [f_*'(x_i-) - f_*'(x_i+)]\eta(x_i) + f_*'(1)\eta(1) - f_*'(0)\eta(0) \\
& = \sum_{i = 1}^{n - 1} [f_*'(x_i-) - f_*'(x_i+)]\eta(x_i)
\end{split}\]
与下式比较各项系数
\[-2 \sum_{i = 1}^{n - 1} \frac{h_{i} + h_{i + 1}}{2}[\tilde y_i - f_*(x_i)]\eta(x_i) + 2\alpha \int_{0}^{1} f_*'(x)\eta'(x) \mathrm{d}x = 0\]
得
\[\alpha[f_*'(x_i-) - f_*'(x_i+)] = \frac{h_{i} + h_{i + 1}}{2}[\tilde y_i - f_*(x_i)]\]
则$f_*$为满足此条件的分段线性函数,且满足$f(0) = \tilde y_0, f(1) = \tilde y_n$.
\end{document}