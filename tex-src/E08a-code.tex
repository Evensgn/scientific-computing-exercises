\documentclass[12pt, a4paper]{article}
\usepackage{ctex}

\usepackage{float}
\usepackage[super, square]{natbib}
\usepackage[margin = 1in]{geometry}
\usepackage{
  color,
  clrscode,
  amssymb,
  ntheorem,
  amsmath,
  listings,
  fontspec,
  xcolor,
  supertabular,
  multirow
}
\definecolor{bgGray}{RGB}{36, 36, 36}
\usepackage[
  colorlinks,
  linkcolor=bgGray,
  anchorcolor=blue,
  citecolor=black
]{hyperref}
\newfontfamily\courier{Courier}

\theoremstyle{margin}
\theorembodyfont{\normalfont}

\newtheorem{theorem}{Theorem}
\newtheorem{definition}[theorem]{Definition}
\newtheorem{example}[theorem]{Example}

\newcommand{\st}{\text{s.t.}}
\newcommand{\mn}{\mathnormal}
\newcommand{\tbf}{\textbf}
\newcommand{\fl}{\mathnormal{fl}}
\newcommand{\f}{\mathnormal{f}}
\newcommand{\g}{\mathnormal{g}}
\newcommand{\R}{\mathbf{R}}
\newcommand{\Q}{\mathbf{Q}}
\newcommand{\JD}{\textbf{D}}
\newcommand{\rd}{\mathrm{d}}
\newcommand{\str}{^*}
\newcommand{\vep}{\varepsilon}
\newcommand{\lhs}{\text{L.H.S}}
\newcommand{\rhs}{\text{R.H.S}}
\newcommand{\con}{\text{Const}}
\newcommand{\oneton}{1,\,2,\,\dots,\,n}
\newcommand{\aoneton}{a_1a_2\dots a_n}
\newcommand{\xoneton}{x_1,\,x_2,\,\dots,\,x_n}


\title{E08a 编程作业解答}
\author{姓名:范舟 \quad 学号:516030910574}
\date{}

\begin{document}
\lstset{numbers=left,
  basicstyle=\scriptsize\courier,
  numberstyle=\tiny\courier\color{red!89!green!36!blue!36},
  language=Matlab,
  breaklines=true,
  keywordstyle=\color{blue!70},commentstyle=\color{red!50!green!50!blue!50},
  morekeywords={},
  stringstyle=\color{purple},
  frame=shadowbox,
  rulesepcolor=\color{red!20!green!20!blue!20}
}
\maketitle

\paragraph{注意:} 
\begin{enumerate}
\item 程序在文档中也要粘贴,同时把代码和该文档放在同一个文件夹中打包发给我.(建议多个同学或整个班级一起打包;邮箱: terenceyuyue@sjtu.edu.cn)
\item 该文档不需打印,只收电子版.
\end{enumerate}

\section{问题1的解答}
求下列方程的实根
\paragraph{(1)} $x^2 - 3x + 2 - e^x = 0$
\paragraph{(2)} $x^3 + 2x^2 + 10x - 20 = 0$

\paragraph{1.1} \textbf{设计不动点迭代}
\par 对方程(1)和(2),分别设计不动点迭代,即给出相应的$\varphi(x)$,并说明其收敛性.
\paragraph{方程(1)} 首先分析方程(1)的根所在的大致区间,设$f(x) = x^2 - 3x + 2 - e^x$,分析可知$f'(x) = 2x - 3 - e^x < 0$在$R$上恒成立,因此$f(x)$在$R$上单调递减,$f(x) = 0$至多有一个实根,又有
\[f(0) = 1 > 0, \quad f(1) = -e < 0\]
因此$f(x) = 0$存在唯一的实根$x_*$,且$x_* \in (0, 1)$.\\
令迭代公式为
\[x = \varphi(x) = \frac{1}{3} (x^2 - e^x + 2)\]
则$\varphi'(x) = \frac{1}{3} (2x - e^x)$,在$(0, 1)$上,$\varphi'(x)$连续,且有
\[|\varphi'(x)| < \max (|\varphi'(0), \varphi'(1)|) = \frac{1}{3} < 1\]
因此此迭代公式局部收敛.且$\varphi'(x) \ne 0$在$(0, 1)$上成立,因此此迭代公式为线性收敛.

\paragraph{方程(2)} 首先分析方程(2)的根所在的大致区间,设$f(x) = x^3 + 2x^2 + 10x - 20$,分析可知$f'(x) = 3x^2 + 4x + 10 > 0$在$R$上恒成立,因此$f(x)$在$R$上单调递增,$f(x) = 0$至多有一个实根,又有
\[f(1) = -7 < 0, \quad f(2) = 16 > 0\]
因此$f(x) = 0$存在唯一的实根$x_*$,且$x_* \in (1, 2)$.\\
令迭代公式为
\[x = \varphi(x) = x - \frac{1}{100} (x^3 + 2x^2 + 10x - 20)\]
则$\varphi'(x) = 1 - \frac{1}{100} (3x^2 + 4x + 10)$,在$(1, 2)$上,$\varphi'(x)$连续,且有
\[|\varphi'(x)| < \max (|\varphi'(1), \varphi'(2)|) = \frac{83}{100} < 1\]
因此此迭代公式局部收敛.且$\varphi'(x) \ne 0$在$(1, 2)$上成立,因此此迭代公式为线性收敛.

\paragraph{1.2} \textbf{Steffensen加速迭代法}
\paragraph{1)} 对之前设计的不动点迭代格式,编写直接迭代的程序,命名为direct\_iteration.m,迭代停止准则为$|x_{k} - x_{k - 1}| < 10^{-8}$.
\par 为了方便,以下将各类迭代方式都实现为接口统一的函数,函数接受3个参数x0,phi和eps,其中x0为迭代初值,phi为在Matlab脚本中实现的迭代函数$\varphi(x)$的函数句柄,eps为迭代停止准则中设定的误差界;并在test\_room.m中定义了题目中使用的迭代函数$\varphi(x)$,并调用实现好的迭代方法函数进行方程根的求解.
\par 下面是在direct\_iteration.m中实现的直接迭代的函数.
\lstinputlisting{./E08a-code/direct_iteration.m}
\par 下面是在test\_room.m中定义的两个迭代函数以及调用direct\_iteration函数求根的代码.
\begin{lstlisting}
x0 = 1;
eps = 1e-8;
ans1 = direct_iteration(x0, @phi1,eps);
ans2 = direct_iteration(x0, @phi2, eps);
fprintf("%.8f %.8f\n", ans1, ans2);

function y = phi1(x)
    y = (x^2 - exp(x) + 2) / 3;
end

function y = phi2(x)
    y = x - (x^3 + 2 * x^2 + 10 * x - 20) / 100;
end
\end{lstlisting}
\par 运行此脚本,得到方程(1)的根为$0.25753028$,方程(2)的根为$1.36880807$.

\paragraph{2)} 编写Steffensen加速程序,命名为Steffensen.m, 在相同条件下重复方程(1)的计算:给出不同初值下,加速前和加速后的迭代步数(列出表格).
\par 下面是在Steffensen.m中实现的Steffensen加速迭代函数.
\lstinputlisting{./E08a-code/Steffensen.m}
\par 在test\_room.m中调用Steffensen函数求方程(1)的根.
\begin{lstlisting}
ans1_steffensen = Steffensen(x0, @phi1,eps);
fprintf("ans1_steffensen = %.8f\n", ans1_steffensen);
\end{lstlisting}
\par 运行脚本得到方程(1)的根为$0.25753029$.
\par 设定一组不同的初值,分别调用direct\_iteration函数和Steffensen函数求方程(1)的根并记录迭代步数,得到下表的结果.可以明显看出Steffensen加速有效减少了迭代步数.
\begin{table}[H]
\centering
\begin{tabular}{|c|c|c|}
\hline
\multirow{2}{*}{迭代初值} & \multicolumn{2}{c|}{迭代步数} \\ \cline{2-3} 
                      & 加速前         & 加速后         \\ \hline
0                     & 14            & 4           \\ \hline
0.2                   & 13            & 3            \\ \hline
0.3                   & 13            & 3            \\ \hline
0.5                   & 14            & 4            \\ \hline
1                     & 15            & 4            \\ \hline
2                     & 16            & 4            \\ \hline
\end{tabular}
\end{table}

\paragraph{1.3} \textbf{Newton迭代}
\paragraph{1)} 针对方程(2),讨论其实根的范围,并判断其是单根还是重根.
\par 设$f(x) = x^3 + 2x^2 + 10x - 20$,由$b^2 - 4ac$判别式小于$0$可知$f'(x) = 3x^2 + 4x + 10 > 0$在$R$上恒成立,因此$f(x)$在$R$上单调递增,$f(x) = 0$至多有一个实根,且此实根为单根,又有
\[f(1) = -7 < 0, \quad f(2) = 16 > 0\]
因此方程(2)存在唯一的实根$x_*$,$x_* \in (1, 2)$,且$x_*$为单根.

\paragraph{2)} 编写基于课本P227式(4.14)的Newton迭代程序,命名为Newton\_iteration.m, 迭代停止准则同前.
\par 课本P227式(4.14)的迭代函数为
\[\varphi(x) = x - \frac{f(x)f'(x)}{[f'(x)]^2 - f(x)f''(x)}\]
\par 下面是在Newton\_iteration.m中实现的基于上式的Newton迭代过程,依照上式求出$\varphi(x)$后,可利用上面实现的direct\_iteration函数求出根.这段代码求解了方程(1)和方程(2)的根.
\begin{lstlisting}
x0 = 1;
eps = 1e-8;
ans1 = direct_iteration(x0, @(x) Newton_phi(@f1, @d_f1, @d2_f1, x), eps);
ans2 = direct_iteration(x0, @(x) Newton_phi(@f2, @d_f2, @d2_f2, x), eps);
fprintf("ans1 = %.8f, ans2 = %.8f\n", ans1, ans2);

function y = Newton_phi(f, d_f, d2_f, x)
    y = x - (f(x) * d_f(x)) / (d_f(x)^2 - f(x) * d2_f(x)); 
end
\end{lstlisting}
上述代码使用了方程(1)与方程(2)的相关函数及导函数,如下
\begin{lstlisting}
function y = f1(x)
    y = x^2 - 3 * x + 2 - exp(x);
end

function y = d_f1(x)
    y = 2 * x - 3 - exp(x);
end

function y = d2_f1(x)
    y = 2 - exp(x);
end

function y = f2(x)
    y = x^3 + 2 * x^2 + 10 * x - 20; 
end

function y = d_f2(x)
    y = 3 * x^2 + 4 * x + 10;
end

function y = d2_f2(x)
    y = 6 * x + 4;
end
\end{lstlisting}
\par 运行此脚本得到方程(1)的根为$0.25753029$,方程(2)的根为$1.36880811$.

\paragraph{3)} 在相同条件下,比较Steffensen加速法与Newton迭代法的迭代步数.
\paragraph{方程(1)} 使用不同的初值,分别使用两种方法求解方程(1)的根,得到迭代步数如下表.
\begin{table}[H]
\centering
\begin{tabular}{|c|c|c|}
\hline
\multirow{2}{*}{迭代初值} & \multicolumn{2}{c|}{迭代步数} \\ \cline{2-3} 
                      & Steffensen加速法        & Newton迭代法         \\ \hline
0                     & 4            & 4           \\ \hline
0.01                   & 3            & 4            \\ \hline
0.2                   & 3            & 4            \\ \hline
0.3                   & 3            & 3            \\ \hline
0.5                   & 4            & 4            \\ \hline
1                     & 4            & 5            \\ \hline
2                     & 4            & 8            \\ \hline
5                     & 5            & 8            \\ \hline
\end{tabular}
\end{table}

\paragraph{方程(2)} 使用不同的初值,分别使用两种方法求解方程(2)的根,得到迭代步数如下表.
\begin{table}[H]
\centering
\begin{tabular}{|c|c|c|}
\hline
\multirow{2}{*}{迭代初值} & \multicolumn{2}{c|}{迭代步数} \\ \cline{2-3} 
                      & Steffensen加速法        & Newton迭代法         \\ \hline
0.01                   & 6            & 5            \\ \hline
0.5                   & 5            & 5            \\ \hline
0.7                   & 5            & 5            \\ \hline
1.5                     & 4            & 4            \\ \hline
2                     & 5            & 5            \\ \hline
5                     & 6            & 6            \\ \hline
7                     & 6            & 8            \\ \hline
\end{tabular}
\end{table}

\paragraph{4)} 选做:若方程有复根,应如何求解?可能的话,求出方程(2)的一个复根(如果有的话).
\paragraph{答} 使用Newton迭代法等求解非线性方程根的不动点迭代法,只要将迭代初值设定为在复根附近的复数,迭代将收敛到复根,与求解实根的方法几乎完全相同.
\par 使用上面实现的Newton\_iteration.m,将迭代初值设定为$x_0 = -5 + 5i$,其余代码无需改动,即可直接求得方程(2)的一个复根$-1.6844 + 3.4313i$;若将迭代初值设定为$x_0 = -5 - 5i$,则求得方程(2)的另一个复根$-1.6844 - 3.4313i$.
\par 事实上,除一个实根外,方程(2)仅有这两个复根.


\section{问题2的解答}
\par 多项式求根是一个病态问题,考虑多项式
\[p(x) = (x - 1)(x - 2)\dots(x - 10) = a_0 + a_1x + \dots + a_9x^9 + x^{10}\]
求解扰动方程$p(x) + \varepsilon x^9 = 0$.

\paragraph{2.1} \textbf{病态性分析}
\par 按照展开式给定系数,对$\varepsilon = 10^{-6}, 10^{-8}, 10^{-10}$,用MATLAB求根函数计算扰动方程的根,分析$\varepsilon$对根的影响(注意MATLAB多项式的存储方式).
\par 在Matlab中使用下面的代码调用expand函数展开多项式$p(x)$.
\begin{lstlisting}
syms x
p = (x - 1) * (x - 2) * (x - 3) * (x - 4) * (x - 5) * ...
    (x - 6) * (x - 7) * (x - 8) * (x - 9) * (x - 10);
expand_p = expand(p)
\end{lstlisting}
\par 得$p(x)$的系数如下
\[\begin{split}
p(x) & = (x - 1)(x - 2)\dots(x - 10) \\
& = x^{10} - 55 x^9 + 1320 x^8 - 18150 x^7 + \dots + 12753576 x^2 - 10628640 x + 3628800
\end{split}\]
\par 下面是在poly\_root.m中计算扰动方程的根的代码.
\begin{lstlisting}
eps = 1e-10;
p = [1 (-55+eps) 1320 -18150 157773 -902055 3416930 -8409500 12753576 -10628640 3628800];
ans = roots(p);
for i = 1:9
    fprintf("%.8f, ", ans(i));
end
fprintf("%.8f\n", ans(10));
\end{lstlisting}
得到的计算结果如下:
\begin{lstlisting}
# eps = 1e-6:
roots = 9.99722949, 9.00954409, 7.98668735, 7.00940116, 5.99651655, 5.00067909, 3.99993933, 3.00000195, 1.99999999, 1.00000000

# eps = 1e-8:
roots = 9.99997244, 9.00009608, 7.99986685, 7.00009342, 5.99996501, 5.00000678, 3.99999939, 3.00000002, 2.00000000, 1.00000000

# eps = 1e-10
roots = 9.99999972, 9.00000096, 7.99999867, 7.00000093, 5.99999965, 5.00000007, 3.99999999, 3.00000000, 2.00000000, 1.00000000
\end{lstlisting}
可以看出,$\varepsilon$较小时,若扰动量是多项式次数较高项的系数,也能对方程的根产生相当的影响,因此求解多项式的根是病态问题.

\paragraph{2.2} \textbf{解决方案思考}
\par 选做:可能的话提出一个稳定化求解的策略(可以查阅文献).
\paragraph{答} 对于解决这一问题本身的病态性,我并没有很好的想法...或许可以在求解实际问题时,尽量避免对多项式高次项系数的直接测量,或考虑使用提高测量精度的方法减小系数的扰动.

\end{document}