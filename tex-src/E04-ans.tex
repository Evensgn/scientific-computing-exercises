\documentclass[12pt, a4paper]{article}
\usepackage{ctex}

\usepackage{float}
\usepackage[super, square]{natbib}
\usepackage[margin = 1in]{geometry}
\usepackage{
  color,
  clrscode,
  amssymb,
  ntheorem,
  amsmath,
  listings,
  fontspec,
  xcolor,
  supertabular,
  multirow
}
\definecolor{bgGray}{RGB}{36, 36, 36}
\usepackage[
  colorlinks,
  linkcolor=bgGray,
  anchorcolor=blue,
  citecolor=black
]{hyperref}
\newfontfamily\courier{Courier}

\theoremstyle{margin}
\theorembodyfont{\normalfont}

\newtheorem{theorem}{Theorem}
\newtheorem{definition}[theorem]{Definition}
\newtheorem{example}[theorem]{Example}

\newcommand{\st}{\text{s.t.}}
\newcommand{\mn}{\mathnormal}
\newcommand{\tbf}{\textbf}
\newcommand{\fl}{\mathnormal{fl}}
\newcommand{\f}{\mathnormal{f}}
\newcommand{\g}{\mathnormal{g}}
\newcommand{\R}{\mathbf{R}}
\newcommand{\Q}{\mathbf{Q}}
\newcommand{\JD}{\textbf{D}}
\newcommand{\rd}{\mathrm{d}}
\newcommand{\str}{^*}
\newcommand{\vep}{\varepsilon}
\newcommand{\lhs}{\text{L.H.S}}
\newcommand{\rhs}{\text{R.H.S}}
\newcommand{\con}{\text{Const}}
\newcommand{\oneton}{1,\,2,\,\dots,\,n}
\newcommand{\aoneton}{a_1a_2\dots a_n}
\newcommand{\xoneton}{x_1,\,x_2,\,\dots,\,x_n}


\title{科学计算 Exercise 4}
\author{范舟\\516030910574\\致远学院2016级ACM班}
\date{}

\begin{document}

\lstset{numbers=left,
  basicstyle=\scriptsize\courier,
  numberstyle=\tiny\courier\color{red!89!green!36!blue!36},
  language=C++,
  breaklines=true,
  keywordstyle=\color{blue!70},commentstyle=\color{red!50!green!50!blue!50},
  morekeywords={},
  stringstyle=\color{purple},
  frame=shadowbox,
  rulesepcolor=\color{red!20!green!20!blue!20}
}
\maketitle

8. 解:
\[\begin{split}
\varphi_0(x)&=1 \\
\varphi_1(x)&=x-\frac{(x,\varphi_0(x))}{(\varphi_0(x),\varphi_0(x))}\varphi_0(x)=x-\frac{\int_{-1}^{1}(1+x^2)x\mathrm{d}x}{\int_{-1}^{1}(1+x^2)\mathrm{d}x}=x \\
\varphi_2(x)&=x^2-\frac{(x^2,\varphi_0(x))}{(\varphi_0(x),\varphi_0(x))}\varphi_0(x)-\frac{(x^2,\varphi_1(x))}{(\varphi_1(x),\varphi_1(x))}\varphi_1(x) \\
&=x^2-\frac{\int_{-1}^{1}(1+x^2)x^2\mathrm{d}x}{\int_{-1}^{1}(1+x^2)\mathrm{d}x}-\frac{\int_{-1}^{1}(1+x^2)x^3\mathrm{d}x}{\int_{-1}^{1}(1+x^2)x^2\mathrm{d}x}x \\
&=x^2-\frac{2}{5} \\
\varphi_3(x)&=x^3-\frac{(x^3,\varphi_0(x))}{(\varphi_0(x),\varphi_0(x))}\varphi_0(x)-\frac{(x^3,\varphi_1(x))}{(\varphi_1(x),\varphi_1(x))}\varphi_1(x)-\frac{(x^3,\varphi_2(x))}{(\varphi_2(x),\varphi_2(x))}\varphi_2(x) \\
&=x^3-\frac{\int_{-1}^{1}(1+x^2)x^3\mathrm{d}x}{\int_{-1}^{1}(1+x^2)\mathrm{d}x}-\frac{\int_{-1}^{1}(1+x^2)x^4\mathrm{d}x}{\int_{-1}^{1}(1+x^2)x^2\mathrm{d}x}x-\frac{\int_{-1}^{1}(1+x^2)(x^2-\frac{2}{5})x^3\mathrm{d}x}{\int_{-1}^{1}(1+x^2)(x^2-\frac{2}{5})^2\mathrm{d}x}(x^2-\frac{2}{5})  \\
&=x^3-\frac{9}{14}x
\end{split}\]
\newline

10. 证明:令$x=cos\theta$,则有
\[\begin{split}
\int_{-1}^{1}\frac{[T_n(x)]^2}{\sqrt{1-x^2}}\mathrm{d}x&=\int_{\pi}^{0}\frac{\cos^2 n\theta}{\sin\theta}\mathrm{d}(\cos\theta)=\int_{\pi}^{0}\frac{\cos^2 n\theta}{\sin\theta}(-\sin \theta)\mathrm{d}\theta \\
&=\int_{0}^{\pi}\cos^2 n\theta\mathrm{d}\theta=\int_{0}^{n\pi}\cos^2 t\mathrm{d}\frac{t}{n} \\
&=\frac{1}{n}\int_{0}^{n\pi}\cos^2 t\mathrm{d}t=\int_{0}^{\pi}\cos^2 t\mathrm{d}t \\
&=\int_{0}^{\pi}\frac{1-\cos 2t}{2}\mathrm{d}t=\int_{0}^{2\pi}\frac{1-\cos u}{4}\mathrm{d}u=\frac{\pi}{2}
\end{split}\]
\newline

11. 解:$T_3(x)$在$[-1,1]$上的零点为
\[x_1=\cos\frac{\pi}{6}=\frac{\sqrt{3}}{2},\ x_2=\cos\frac{\pi}{2}=0,\ x_3=\cos\frac{5\pi}{6}=-\frac{\sqrt{3}}{2}\]
以$x_1,x_2,x_3$做插值点,$f(x)=e^x$的二次插值多项式为
\[\begin{split}
P_2(x)&=f(x_3)+f[x_3,x_2](x-x_3)+f[x_3,x_2,x_1](x-x_3)(x-x_2) \\
&=e^{-\frac{\sqrt{3}}{2}}+\frac{1-e^{-\frac{\sqrt{3}}{2}}}{\frac{\sqrt{3}}{2}}(x+\frac{\sqrt{3}}{2})+\frac{2(e^{\frac{\sqrt{3}}{2}}+e^{-\frac{\sqrt{3}}{2}}-2)}{3}(x+\frac{\sqrt{3}}{2})x \\
&=\frac{2}{3}(e^{-\frac{\sqrt{3}}{2}}+e^{\frac{\sqrt{3}}{2}}-2)x^2+\frac{e^{\frac{\sqrt{3}}{2}}-e^{-\frac{\sqrt{3}}{2}}}{\sqrt{3}}x+1
\end{split}\]
\[R_2(x)=f(x)-P_2(x)=\frac{f^{(3)}(\xi)}{3!}\omega_3(x)\]
记
\[M_3=\Vert f^{(3)}(x) \Vert _\infty=\max_{-1\le x \le 1}|f^{(3)}(x)|=e\]
则有
\[\begin{split}
\max_{-1\le x\le1}|R_2(x)|&\le \frac{M_3}{3!}\max_{-1\le x \le 1}|(x-x_1)(x-x_2)(x-x_3)| \\
&=\frac{e}{3!}\frac{1}{2^2}=\frac{e}{24}
\end{split}\]
即误差界为$\frac{e}{24}$.
\newline

12. 解:若取$\Phi=\mathrm{span}\{1,x\}$,令$t=T(x)=ax+b$,使得$T(0)=-1,T(1)=1$.得$t=2x-1$,则$t\in[-1,1]$,且
\[x=\frac{t+1}{2},\ f(x)=g(t)=x^2+3x+2=\frac{1}{4}t^2+2t+\frac{15}{4}\]
设最佳平方逼近多项式为$Q(t)$,则
\[f(t)-Q(t)=\frac{1}{4}\widetilde{P}_2\]
其中$\widetilde{P}_2$为首项系数为$1$的二次Legendre多项式,
\[\widetilde{P}_2=\frac{2!}{4!}\frac{\mathrm{d}^2}{\mathrm{d}t^2}(t^2-1)^2=t^2-\frac{1}{3}\]
则
\[Q(t)=f(t)-\frac{1}{4}\widetilde{P}_2=\frac{1}{4}t^2+2t+\frac{15}{4}-\frac{1}{4}(t^2-\frac{1}{3})=2t+\frac{23}{6}\]
最佳平方逼近多项式为
\[P(x)=Q(t)=4x+\frac{11}{6}\]
若取$\Phi=\mathrm{span}\{1,x,x^2\}$,最佳平方逼近多项式为$f(x)$本身,即
\[P(x)=x^2+3x+2\]

\end{document}