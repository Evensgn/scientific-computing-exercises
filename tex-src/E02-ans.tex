\documentclass[12pt, a4paper]{article}
\usepackage{ctex}

\usepackage{float}
\usepackage[super, square]{natbib}
\usepackage[margin = 1in]{geometry}
\usepackage{
  color,
  clrscode,
  amssymb,
  ntheorem,
  amsmath,
  listings,
  fontspec,
  xcolor,
  supertabular,
  multirow
}
\definecolor{bgGray}{RGB}{36, 36, 36}
\usepackage[
  colorlinks,
  linkcolor=bgGray,
  anchorcolor=blue,
  citecolor=black
]{hyperref}
\newfontfamily\courier{Courier}

\theoremstyle{margin}
\theorembodyfont{\normalfont}

\newtheorem{theorem}{Theorem}
\newtheorem{definition}[theorem]{Definition}
\newtheorem{example}[theorem]{Example}

\newcommand{\st}{\text{s.t.}}
\newcommand{\mn}{\mathnormal}
\newcommand{\tbf}{\textbf}
\newcommand{\fl}{\mathnormal{fl}}
\newcommand{\f}{\mathnormal{f}}
\newcommand{\g}{\mathnormal{g}}
\newcommand{\R}{\mathbf{R}}
\newcommand{\Q}{\mathbf{Q}}
\newcommand{\JD}{\textbf{D}}
\newcommand{\rd}{\mathrm{d}}
\newcommand{\str}{^*}
\newcommand{\vep}{\varepsilon}
\newcommand{\lhs}{\text{L.H.S}}
\newcommand{\rhs}{\text{R.H.S}}
\newcommand{\con}{\text{Const}}
\newcommand{\oneton}{1,\,2,\,\dots,\,n}
\newcommand{\aoneton}{a_1a_2\dots a_n}
\newcommand{\xoneton}{x_1,\,x_2,\,\dots,\,x_n}


\title{科学计算 Exercise 2}
\author{范舟\\516030910574\\致远学院2016级ACM班}
\date{}

\begin{document}

\lstset{numbers=left,
  basicstyle=\scriptsize\courier,
  numberstyle=\tiny\courier\color{red!89!green!36!blue!36},
  language=C++,
  breaklines=true,
  keywordstyle=\color{blue!70},commentstyle=\color{red!50!green!50!blue!50},
  morekeywords={},
  stringstyle=\color{purple},
  frame=shadowbox,
  rulesepcolor=\color{red!20!green!20!blue!20}
}
\maketitle

\section{教材练习 P48-49}

首先证明下面的定理1。

\paragraph{定理1} 设在区间$\left[a,b\right]$上给定$n+1$个点
$$a\leq x_0<x1<\dots<x_n\leq b$$
上的函数值$y_i=f\left(i\right)\left(i=0,1,\dots,n\right)$,求一次数不超过$n$的多项式$P\left(x\right)=a_0+a_1 x+\dots+a_n x^n$,使
$$P\left(x_i\right)=y_i,\ i=0,1,\dots,n$$
则多项式$P(x)$是存在且唯一的。

\paragraph{证明} 由定理条件可得关于多项式系数$a_0,a_1,\dots,a_n$的$n+1$元线性方程组
$$\left\{\begin{matrix}
a_0 + a_1 x_0 + \dots + a_n x_0^n = y_0 \\ 
a_0 + a_1 x_1 + \dots + a_n x_1^n = y_1 \\ 
\vdots  \\ 
a_0 + a_1 x_n + \dots + a_n x_n^n = y_n
\end{matrix}\right.$$
此方程的系数矩阵$A$为一个Vandermonde矩阵,因$x_i\left(i=0,1,\dots,n\right)$互异,其行列式
$$\det A=\prod_{i,j=0,\ i>j}^{n}\left(x_i-x_j\right) \ne 0$$ 
因此线性方程组的解$a_0,a_1,\dots,a_n$存在且唯一,证毕。
\newline

4.

(1) 等式两边均为次数不超过$n$的多项式,且在$x_j\left(j=0,1,\dots,n\right)$处的函数值相等(都等于$x_j^k$),由定理1,等式两边的多项式相等。

(2) 等式两边均为次数不超过$n$的多项式,$x_j\left(j=0,1,\dots,n\right)$处的函数值相等(都等于$0$),由定理1,等式两边的多项式相等。
\newline

8. 若$f\left(x\right)$在$\left[a,b\right]$上存在$n$阶导数,且节点$x_0,x_1,\dots,x_n\in \left[a,b\right]$,则满足
$$f\left[x_0,x_1,\dots,x_n\right]=\frac{f^{\left(n\right)}\left(\xi\right)}{n!},\ \xi\in \left[a,b\right]$$
由此得
$$f\left[2^0,2^1,\dots,2^7\right]=\frac{f^{\left(7\right)}\left(\xi\right)}{7!}\left(\xi \in \left[2^0,2^7\right]\right) = 1$$
$$f\left[2^0,2^1,\dots,2^8\right]=\frac{f^{\left(8\right)}\left(\xi\right)}{8!}\left(\xi \in \left[2^0,2^8\right]\right) = 0$$
\newline

9. 
\[\begin{split}
	\Delta\left(f_k g_k\right)&=f_{k+1} g_{k+1} - f_k g_k = f_{k+1} g_{k+1} + f_{k} g_{k+1} - f_{k} g_{k+1} - f_k g_k \\
	&= f_k\left(g_{k+1}-g_k\right) + g_{k+1} \left(f_{k+1}-f_{k}\right)=f_k \Delta g_k+g_{k+1}\Delta f_k
\end{split}\]
\newline 

10. 
\[\begin{split}
	\sum_{k=0}^{n-1}f_k\Delta g_k &= \sum_{k=0}^{n-1}f_k g_{k+1} - \sum_{k=0}^{n-1}f_k g_{k} \\
	&=\sum_{k=0}^{n-2}f_k g_{k+1} + f_{n-1} g_n - \sum_{k=1}^{n-1}f_k g_{k} - f_0 g_0 \\
	&=f_n g_n - f_0 g_0 + \left(f_{n-1} - f_n\right) g_n + \sum_{k=0}^{n-2}f_k g_{k+1} - \sum_{k=1}^{n-1}f_k g_{k} \\
	&= f_n g_n - f_0 g_0 - \Delta f_{n-1} g_n + \sum_{k=0}^{n-2}f_k g_{k+1} - \sum_{k=0}^{n-2}f_{k+1} g_{k+1} \\
  &= f_n g_n - f_0 g_0 - \sum_{k=0}^{n-1} g_{k+1} \Delta f_k \\
\end{split}\]
\newline 

12. 因$x_1,x_2,\dots,x_n$是$f\left(x\right)$的$n$个不同实根,不妨设$f\left(x\right)=A\left(x-x_1\right)\dots\left(x-x_n\right)$。因此有
\[f'\left(x_j\right)=A\prod_{i\ne j}\left(x-x_i\right)\]
则有
\[\sum_{j=1}^{n}\frac{x_j^k}{f'\left(x_j\right)}=\sum_{j=1}^{n}\frac{x_j^k}{A\prod_{i\ne j}\left(x-x_i\right)}\]
令$g_k\left(x\right)=x^k$,则
\[\begin{split}
\sum_{j=1}^{n}\frac{x_j^k}{f'\left(x_j\right)}&=\frac{1}{A} g_k\left[x_1,x_2,\dots,x_n\right]\\
&=\frac{1}{A} \frac{g^{\left(n-1\right)\left(\xi\right)}}{\left(n-1\right)!}\ \left(\xi \in \left[\min x_j, \max x_j\right],\ j=1,2,\dots,n\right)
\end{split}\]
因此,当$0\leq k\leq n-2$时,$\sum_{j=1}^{n}\frac{x_j^k}{f'\left(x_j\right)}=0$;
当$k=n-1$时,$\sum_{j=1}^{n}\frac{x_j^k}{f'\left(x_j\right)}=\frac{1}{A}=a_n^{-1}$。

\section{补充练习}

1. 已知对于$k=0,1,2,\dots,n$成立$f\left(k\right)-\frac{k}{k+1}=0$,设$g\left(x\right)=f\left(x\right)\left(x+1\right)-x$,则$g\left(x\right)$为一个次数不超过$n+1$的多项式,且$k=0,1,2,\dots,n$为$g\left(x\right)$的$n+1$个零点。可设$g\left(x\right)=Ax\left(x-1\right)\dots\left(x-n\right)$。又知$g\left(-1\right)=1$,代入得
\[g\left(-1\right)=-A\left(-1-1\right)\dots\left(-1-n\right)=1\ \Rightarrow\ A=\frac{\left(-1\right)^{n+1}}{\left(n+1\right)!}\]
因此
\[g\left(x\right)=\frac{\left(-1\right)^{n+1}}{\left(n+1\right)!}x\left(x-1\right)\dots\left(x-n\right)\]
\[f\left(x\right)=\frac{g\left(x\right)+x}{x+1}=\frac{\frac{\left(-1\right)^{n+1}}{\left(n+1\right)!}x\left(x-1\right)\dots\left(x-n\right)+x}{x+1}\]
\newline

2. 首先考虑极端情形,若相邻两点之间的间距均为$h$,即
$$x_{i+1}-x_i=h\left(i=0,1,2,\dots,n-1\right)$$
下面首先证明此情形下的结论:\\
由于结论只与$x_k$之间的相对位置有关,所以不妨设$x_0 = 0$。
根据齐次性,不妨设$h = 1$(否则令$y = \frac{x}{h}$即可)。因此只需证明
\[
  x(x-1)\cdots(x-n) \le \frac{n!}{4}
\]
对于$n=1$的情况,根据均值不等式,成立,
\[
  |x(x-1)| = x(1-x) \le \frac{1}{4}.
\]
假设对于$n$的情况成立,考虑$n+1$的情况。若$x\in[0, n]$,则
\[
  |\left(x-x_0\right)\left(x-x_1\right)\dots\left(x-x_{n+1}\right)| \le \frac{n!}{4}\times(n+1-x) \le \frac{(n+1)!}{4}.
\]
若$x\in[n, n+1]$,结合$n=1$的情况,成立
\[
  |\left(x-x_0\right)\left(x-x_1\right)\dots\left(x-x_{n+1}\right)| \le (n+1)n(n-1)\cdots\frac{1}{4} = \frac{(n+1)!}{4}
\]
因此,相邻两点之间的间距均为$h$时结论成立。\\
对于一般情况,$x_{i+1}-x_i\leq h\left(i=0,1,2,\dots,n-1\right)$,对于任意$x\in \left[x_0,x_n\right]$,不妨设$x\in\left[x_{k},x_{k+1}\right]$,则存在$x_0',x_1',\dots,x_{k}'$与$x_{k+1}',x_{k+2}',\dots,x_{n}'$满足:
\[\begin{split}
x_i'<x_i\ &\left(i=0,1,\dots,k\right) \\
x_i'>x_i\ &\left(i=k+1,k+2,\dots,n\right) \\
x_{i+1}'-x_i'=h\ &\left(i=0,1,2,\dots,n-1\right)
\end{split}\]
则有$|x-x_i|\leq|x-x_i'|\ \left(i=0,1,2,\dots,n\right)$
\[|\left(x-x_0\right)\left(x-x_1\right)\dots\left(x-x_n\right)|\leq|\left(x-x_0'\right)\left(x-x_1'\right)\dots\left(x-x_n'\right)|\leq \frac{n!h^{n+1}}{4}
\]

\end{document}