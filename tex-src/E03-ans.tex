\documentclass[12pt, a4paper]{article}
\usepackage{ctex}

\usepackage{listings}
\usepackage{xcolor}
\lstset{
    columns=fixed,       
    numbers=left,                                        % 在左侧显示行号
    frame=none,                                          % 不显示背景边框
    backgroundcolor=\color[RGB]{245,245,244},            % 设定背景颜色
    keywordstyle=\color[RGB]{40,40,255},                 % 设定关键字颜色
    numberstyle=\footnotesize\color{darkgray},           % 设定行号格式
    commentstyle=\it\color[RGB]{0,96,96},                % 设置代码注释的格式
    stringstyle=\rmfamily\slshape\color[RGB]{128,0,0},   % 设置字符串格式
    showstringspaces=false,                              % 不显示字符串中的空格
    language=Matlab,                                        % 设置语言
}

\usepackage{float}
\usepackage[super, square]{natbib}
\usepackage[margin = 1in]{geometry}
\usepackage{
  color,
  clrscode,
  amssymb,
  ntheorem,
  amsmath,
  listings,
  fontspec,
  xcolor,
  supertabular,
  multirow
}
\definecolor{bgGray}{RGB}{36, 36, 36}
\usepackage[
  colorlinks,
  linkcolor=bgGray,
  anchorcolor=blue,
  citecolor=black
]{hyperref}
\newfontfamily\courier{Courier}

\theoremstyle{margin}
\theorembodyfont{\normalfont}

\newtheorem{theorem}{Theorem}
\newtheorem{definition}[theorem]{Definition}
\newtheorem{example}[theorem]{Example}

\newcommand{\st}{\text{s.t.}}
\newcommand{\mn}{\mathnormal}
\newcommand{\tbf}{\textbf}
\newcommand{\fl}{\mathnormal{fl}}
\newcommand{\f}{\mathnormal{f}}
\newcommand{\g}{\mathnormal{g}}
\newcommand{\R}{\mathbf{R}}
\newcommand{\Q}{\mathbf{Q}}
\newcommand{\JD}{\textbf{D}}
\newcommand{\rd}{\mathrm{d}}
\newcommand{\str}{^*}
\newcommand{\vep}{\varepsilon}
\newcommand{\lhs}{\text{L.H.S}}
\newcommand{\rhs}{\text{R.H.S}}
\newcommand{\con}{\text{Const}}
\newcommand{\oneton}{1,\,2,\,\dots,\,n}
\newcommand{\aoneton}{a_1a_2\dots a_n}
\newcommand{\xoneton}{x_1,\,x_2,\,\dots,\,x_n}


\title{科学计算 Exercise 3}
\author{范舟\\516030910574\\致远学院2016级ACM班}
\date{}

\begin{document}

\lstset{
	numbers=left,
	basicstyle=\scriptsize\courier,
	numberstyle=\tiny\courier\color{red!89!green!36!blue!36},
	language=Matlab,
	breaklines=true,
	keywordstyle=\color{blue!70},commentstyle=\color{red!50!green!50!blue!50},
	morekeywords={},
	stringstyle=\color{purple},
	frame=shadowbox,
	rulesepcolor=\color{red!20!green!20!blue!20}
}
\maketitle

\section{教材练习 P49-50}

13. 解:由于次数不超过三次的多项式$P\left(x\right)$通过点$\left(x_0, f\left(x_0\right)\right),\left(x_1, f\left(x_1\right)\right)$,可设
\[P\left(x\right)=f\left(x_0\right)+f\left[x_0,x_1\right]\left(x-x_0\right)+A\left(x-x_0\right)\left(x-x_1\right)+B\left(x-x_0\right)^2\left(x-x_1\right)\]
又有
\[\begin{split}
	P'\left(x_0\right)&=f\left[x_0,x_1\right]+A\left(x_0-x_1\right)=f'\left(x_0\right) \\
	P''\left(x_0\right)&=2A+2B\left(x_0-x_1\right)=f''\left(x_0\right)
\end{split}\]
解得
\[\begin{split}
	A&=\frac{f'\left(x_0\right)-f\left[x_0,x_1\right]}{x_0-x_1} \\
	&=\frac{f'\left(x_0\right)\left(x_0-x_1\right)+f\left(x_1\right)-f\left(x_0\right)}{\left(x_0-x_1\right)^2} \\
	B&=\frac{f''\left(x_0\right)\left(x_0-x_1\right)+2f\left[x_0,x_1\right]-2f'\left(x_0\right)}{2\left(x_0-x_1\right)^2} \\
	&=\frac{f''\left(x_0\right)\left(x_0-x_1\right)^2-2f'\left(x_0\right)\left(x_0-x_1\right)+2f\left(x_0\right)-2f\left(x_1\right)}{2\left(x_0-x_1\right)^3}
\end{split}\]
因此
\[\begin{split}
	P\left(x\right)&=f\left(x_0\right)+\frac{f\left(x_1\right)-f\left(x_0\right)}{x_1-x_0}\left(x-x_0\right) \\
	&+\frac{f'\left(x_0\right)\left(x_0-x_1\right)+f\left(x_1\right)-f\left(x_0\right)}{\left(x_0-x_1\right)^2}\left(x-x_0\right)\left(x-x_1\right) \\
	&+\frac{f''\left(x_0\right)\left(x_0-x_1\right)^2-2f'\left(x_0\right)\left(x_0-x_1\right)+2f\left(x_0\right)-2f\left(x_1\right)}{2\left(x_0-x_1\right)^3}\left(x-x_0\right)^2\left(x-x_1\right)
\end{split}\]
\newline

15. 证明:可设$[x_k,x_{k+1}]$间的插值余项为
\[R_3 (x)=f(x)-P(x)=k(x)(x-x_k)^2(x-x_{k+1})^2\]
其中$k(x)$为待定函数,固定$x\in[x_k,x_{k+1}]$,构造
\[\phi(t)=f(t)-P(t)-k(x)(t-x_k)^2(t-x_{k+1})^2\]
显然有$\phi(x_k)=\phi(x_{k+1})=\phi(x)=0$,且$\phi'(x_k)=\phi'(x_{k+1})=0$,因此在$\phi(t)$在$[x_k,x_{k+1}]$中至少有$5$个零点(二重根算两个).反复应用罗尔定理,可得$\phi^{(4)}(t)$在$(x_k,x_{k+1})$中至少有一个零点,不妨设其为$\xi$,则有
\[\phi^{(4)}(\xi)=f^{(4)}(\xi)-4!k(x)=0\]
即得到
\[\begin{split}
	k(x)&=\frac{f^{(4)}(\xi)}{4!},\ \xi\in(x_k,x_{k+1}) \\
	R_3 (x)&=\frac{f^{(4)}(\xi)}{4!}(x-x_k)^2(x-x_{k+1})^2
\end{split}\]
设$h_k=x_{k+1}-x_k,h=\max h_k,M=\sup f^{(4)}$,则分段三次埃尔米特插值的误差
\[R_3 (x)=\frac{f^{(4)}(\xi)}{4!}(x-x_k)^2(x-x_{k+1})^2 \le \frac{M}{4!}\frac{h^4}{16}=\frac{Mh^4}{384}\]
因此分段三次埃尔米特插值的误差限为$\frac{Mh^4}{384}$.
\newline

2. 画出的插值图像如下图.
\begin{figure}[H]
  \begin{center}
    \includegraphics[height=14cm]{ex3_pic1.png}
  \end{center}
\end{figure}
Matlab代码:
\begin{lstlisting}
x_10 = -1 : (1 - -1) / 10 : 1;
x_20 = -1 : (1 - -1) / 20 : 1;
f_10 = f(x_10);
f_20 = f(x_20);

x_plot = -1 : (1 - -1) / 500 : 1;
f_plot = f(x_plot);

subplot(2, 2, 1);
plot(x_plot, f_plot, 'b', 'LineWidth', 1);
hold;
y1_plot = newton(x_10, f_10, x_plot);
plot(x_plot, y1_plot, 'r', 'LineWidth', 1);
title('n = 10, Polynomial');
box off;

subplot(2, 2, 2);
plot(x_plot, f_plot, 'b', 'LineWidth', 1);
hold;
y2_plot = newton(x_20, f_20, x_plot);
plot(x_plot, y2_plot, 'r', 'LineWidth', 1);
title('n = 20, Polynomial');
box off;

subplot(2, 2, 3);
plot(x_plot, f_plot, 'b', 'LineWidth', 1);
hold;
y1_plot = spline(x_10, f_10, x_plot);
plot(x_plot, y1_plot, 'r', 'LineWidth', 1);
title('n = 10, Cubic Spline');
box off;

subplot(2, 2, 4);
plot(x_plot, f_plot, 'b', 'LineWidth', 1);
hold;
y2_plot = spline(x_20, f_20, x_plot);
plot(x_plot, y2_plot, 'r', 'LineWidth', 1);
title('n = 20, Cubic Spline');
box off;

function ret = f(x)
    ret = 1 ./ (1 + 25 * x .^ 2);
end

function yq = newton(x, y, xq)
    n = length(x);
    d(:, 1) = y;
    for i = 2 : n
        dx(i : n) = x(i : n) - x(1 : n - i + 1);
        d(i : n, i) = (d(i : n, i - 1) - d(i - 1 : n - 1, i - 1)) ./ dx(i : n)';
        a(i) = d(i, i);
    end
    yq = calc(a, x, xq);
    function y = calc(a, x, xq)
        n = length(x);
        y = 0;
        for i = 1 : n
            t = a(i);
            for j = 1 : i - 1
                t = t .* (xq - x(j));
            end
            y = y + t;
        end
    end
end
\end{lstlisting}

\section{补充练习}

1. 下面的这段Matlab代码中,spline\_zhou()这个函数实现了三种边界条件的三次样条插值,函数的调用方式请见代码开头的注释:
\begin{lstlisting}
% function spline_zhou(), code by Zhou Fan
% Cubic Spline Interpolation under three classes of end conditions
% Usage:
%   condition (1): yq = spline_zhou(x, y, xq, 1, [a, b]);
%   condition (2): yq = spline_zhou(x, y, xq, 2, [a, b]);
%   condition (3): yq = spline_zhou(x, y, xq, 3, []);

function yq = spline_zhou(x, y, xq, cond_type, cond_v)
    n = length(x);
    h(1 : n - 1) = x(2 : n) - x(1 : n - 1);
    mu(2 : n - 1) = h(1 : n - 2) ./ (h(1 : n - 2) + h(2 : n - 1));
    lambda(2 : n - 1) = h(2 : n - 1) ./ (h(1 : n - 2) + h(2 : n - 1));
    d1_f(2 : n) = (y(2 : n) - y(1 : n - 1)) ./ h(1 : n - 1);
    d2_f(3 : n) = (d1_f(3 : n) - d1_f(2 : n - 1)) ./ (h(1 : n - 2) + h(2 : n - 1));
    d(2 : n - 1) = 6 * d2_f(3 : n);
    
    switch cond_type
        case 1
            lambda(1) = 1;
            d(1) = 6 / h(1) * (d1_f(2) - cond_v(1));
            mu(n) = 1;
            d(n) = 6 / h(n - 1) * (cond_v(2) - d1_f(n));
        case 2
            lambda(1) = 0;
            mu(n) = 0;
            d(1) = 2 * cond_v(1);
            d(n) = 2 * cond_v(2);
        case 3
            lambda(n) = h(1) / (h(n - 1) + h(1));
            mu(n) = 1 - lambda(n);
            d(n) = 6 * (d1_f(2) - d1_f(n)) / (h(1) + h(n - 1));
            d(1) = 0;
        otherwise
            throw(MException('Zhou:invalidArgument', 'Invalid argument for function spline_zhou'));
    end
    
    if cond_type <= 2
        A = 2 * eye(n);
        A = A + [zeros(n - 1, 1), lambda(1 : n - 1) .* eye(n - 1); zeros(1, n)];
        A = A + [zeros(1, n); mu(2 : n) .* eye(n - 1), zeros(n - 1, 1)];
    else
        B = 2 * eye(n - 1);
        B = B + [zeros(n - 2, 1), lambda(2 : n - 1) .* eye(n - 2); zeros(1, n - 1)];
        B = B + [zeros(1, n - 1); mu(3 : n) .* eye(n - 2), zeros(n - 2, 1)];
        B(1, n - 1) = mu(2);
        B(n - 1, 1) = lambda(n);
        A = [zeros(1, n); zeros(n - 1, 1), B];
        A(1, 1) = 1;
        A(1, n) = -1;
    end
    
    M = (A \ d')';
    j = discretize(xq, x);
    yq = M(j) .* (x(j + 1) - xq) .^ 3 ./ (6 * h(j)) + M(j + 1) .* (xq - x(j)) .^ 3 ./ (6 * h(j)) ...
        + (y(j) - M(j) .* h(j) .^ 2 ./ 6) .* (x(j + 1) - xq) ./ h(j) ...
        + (y(j + 1) - M(j + 1) .* h(j) .^ 2 ./ 6) .* (xq - x(j)) ./ h(j);
end
\end{lstlisting}
使用spline\_zhou()对上一道题目中的函数分别进行三种条件下的三次样条插值(取$n=10$),得到的图像如下.
\begin{figure}[H]
  \begin{center}
    \includegraphics[height=4cm]{ex3_pic2.png}
  \end{center}
\end{figure}
可以看到,在三种边界条件下,三次样条插值的效果都很好。

\end{document}