\documentclass[12pt, a4paper]{article}
\usepackage{ctex}

\usepackage{float}
\usepackage[super, square]{natbib}
\usepackage[margin = 1in]{geometry}
\usepackage{
  color,
  clrscode,
  amssymb,
  ntheorem,
  amsmath,
  listings,
  fontspec,
  xcolor,
  supertabular,
  multirow
}
\definecolor{bgGray}{RGB}{36, 36, 36}
\usepackage[
  colorlinks,
  linkcolor=bgGray,
  anchorcolor=blue,
  citecolor=black
]{hyperref}
\newfontfamily\courier{Courier}

\theoremstyle{margin}
\theorembodyfont{\normalfont}

\newtheorem{theorem}{Theorem}
\newtheorem{definition}[theorem]{Definition}
\newtheorem{example}[theorem]{Example}

\newcommand{\st}{\text{s.t.}}
\newcommand{\mn}{\mathnormal}
\newcommand{\tbf}{\textbf}
\newcommand{\fl}{\mathnormal{fl}}
\newcommand{\f}{\mathnormal{f}}
\newcommand{\g}{\mathnormal{g}}
\newcommand{\R}{\mathbf{R}}
\newcommand{\Q}{\mathbf{Q}}
\newcommand{\JD}{\textbf{D}}
\newcommand{\rd}{\mathrm{d}}
\newcommand{\str}{^*}
\newcommand{\vep}{\varepsilon}
\newcommand{\lhs}{\text{L.H.S}}
\newcommand{\rhs}{\text{R.H.S}}
\newcommand{\con}{\text{Const}}
\newcommand{\oneton}{1,\,2,\,\dots,\,n}
\newcommand{\aoneton}{a_1a_2\dots a_n}
\newcommand{\xoneton}{x_1,\,x_2,\,\dots,\,x_n}


\title{科学计算 Exercise 09a}
\author{范舟\\516030910574\\致远学院2016级ACM班}
\date{}

\begin{document}
\lstset{numbers=left,
  basicstyle=\scriptsize\courier,
  numberstyle=\tiny\courier\color{red!89!green!36!blue!36},
  language=Matlab,
  breaklines=true,
  keywordstyle=\color{blue!70},commentstyle=\color{red!50!green!50!blue!50},
  morekeywords={},
  stringstyle=\color{purple},
  frame=shadowbox,
  rulesepcolor=\color{red!20!green!20!blue!20}
}
\maketitle

\paragraph{2.} \textbf{证明}
\paragraph{(1)} 因为$\mathbf{A}$是正定矩阵,则有
\[\mathbf{x^TAx} > 0, \quad \forall \mathbf{x} \in \mathbf{R^n} \setminus \{\mathbf{0}\}\]
考虑$\mathbf{R^n}$的单位正交基${\mathbf{e_1}, \mathbf{e_2}, \dots, \mathbf{e_n}}$,其中
\[\mathbf{e_k} = (0, \dots, 0, \mathop{1}\limits_{k}, 0, \dots, 0)^T, \quad k = 1, 2, \dots, n\]
则有
\[a_{ii} = \mathbf{e_i^TAe_i} > 0, \quad i = 1, 2, \dots, n\]
证毕.
\paragraph{(2)} 首先证明$\mathbf{A_2}$是对称矩阵.由(1)知$a_{11} > 0$,经过高斯消去一步后,$\mathbf{A_2} = (a_{ij}^{(2)})_{n - 1}$,有
\[a_{ij}^{(2)} = a_{ij} - \frac{a_{i1}}{a_{11}}a_{1j}\]
由于$\mathbf{A} = (a_{ij})_n$为对称矩阵,有$a_{ij} = a{ji}, \quad i, j = 1, 2, \dots, n$,因此
\[a_{ij}^{(2)} = a_{ij} - \frac{a_{i1}}{a_{11}}a_{1j} = a_{ji} - \frac{a_{j1}}{a_{11}}a_{1i} = a_{ji}^{(2)}, \quad i, j = 2, 3, \dots, n\]
因此$\mathbf{A_2}$为对称矩阵.\\
由于每次初等行变换可看作左乘一初等阵$\mathbf{P_k}$,而高斯消去法的每步操作均为初等行变换,因此有
\[\mathbf{PA}=\begin{bmatrix}
a_{11} & \mathbf{a_1^T}\\  
\mathbf{0} & \mathbf{A_2}
\end{bmatrix}, \quad \mathbf{P} = \mathbf{P_tP_{t-1}} \dots \mathbf{P_1}\]
其中$\mathbf{P}$为每步初等行变换对应的初等阵$\mathbf{P_k}$的乘积,由于初等阵均为可逆矩阵,因此$\mathbf{P}$也是可逆的.
由于$A$是对称矩阵,有
\[\mathbf{PAP^T} = \begin{bmatrix}
a_{11} & \mathbf{0} \\  
\mathbf{0} & \mathbf{A_2}
\end{bmatrix}\]
对于$\mathbf{R^n}$中的任一非零向量$\mathbf{x}$,因为$\mathbf{P}$是可逆的,有$\mathbf{xP^T}$非零,则有
\[\mathbf{x^T}\begin{bmatrix}
a_{11} & \mathbf{0} \\  
\mathbf{0} & \mathbf{A_2}
\end{bmatrix}\mathbf{x} = \mathbf{x^T PAP^Tx} = \mathbf{(xP^T)^TA(xP^T)} > 0\]
因此$\begin{bmatrix}
a_{11} & \mathbf{0} \\  
\mathbf{0} & \mathbf{A_2}
\end{bmatrix}$是正定矩阵,因此其各阶顺序主子式均大于0,又因为$a_{11} > 0$,可得$\mathbf{A_2}$的各阶顺序主子式均大于0,则$\mathbf{A_2}$为正定矩阵,证毕.
\newline

\paragraph{4.} \textbf{解}\\
设$\mathbf{A} = \begin{bmatrix}
a_{11} & a_{12} & \dots & a_{1n} \\ 
a_{21} & a_{22} & \dots & a_{2n} \\ 
\vdots & \vdots &  & \vdots \\ 
a_{n1} & a_{n2} & \dots & a_{nn} 
\end{bmatrix}$,$\mathbf{L} = \begin{bmatrix}
l_{11} &  &  &  \\ 
l_{21} &  l_{22} &  &  \\ 
\vdots & \vdots & \ddots &  \\ 
l_{n1} & l_{n2} & \dots & l_{nn} 
\end{bmatrix}$,$\mathbf{U} = \begin{bmatrix}
1 & u_{12} & \dots & u_{1n} \\ 
 &  1 &  & u_{2n} \\ 
 &  & \ddots &  \vdots\\ 
 &  &  & 1 
\end{bmatrix}$
由$\mathbf{A} = \mathbf{LU}$,得
\[\begin{split}
u_{ii} = 1, & \quad i = 1,2,\dots,n \\
a_{1i} = l_{11}u_{1i}, \quad u_{1i} = \frac{a_{1i}}{l_{11}}, & \quad i = 1,2,\dots,n \\
a_{i1} = l_{i1}u_{11} = l_{i1}, & \quad i = 1,2,\dots,n
\end{split}\]
对于$\mathbf{L}$的第$k$列和$\mathbf{U}$的第$k$行,有
\[\begin{split}
a_{ki} = \sum_{t = 1}^{k}l_{kt}u_{ti}, & \quad i = 1,2,\dots,n \\
u_{ki} = \frac{a_{ki} - \sum_{t = 1}^{k - 1}l_{kt}u_{ti}}{l_{kk}}, & \quad i = 1,2,\dots,n \\
a_{ik} = \sum_{t = 1}^{k}l_{it}u_{tk}, & \quad i = 1,2,\dots,n \\
l_{ik} = \frac{a_{ik} - \sum_{t = 1}^{k - 1}l_{kt}u_{ti}}{u_{kk}} = a_{ik} - \sum_{t = 1}^{k - 1}l_{kt}u_{ti}, & \quad i = 1,2,\dots,n 
\end{split}\]
因此,可以先求出$\mathbf{L}$的第$1$列和$\mathbf{U}$的第$1$行
\[\begin{split}
u_{1i} = \frac{a_{1i}}{l_{11}}, & \quad i = 1,2,\dots,n \\
l_{i1} = a_{i1}, & \quad i = 1,2,\dots,n
\end{split}\]
之后依次递增$k$,按照下面的公式求出$\mathbf{L}$的第$k$列和$\mathbf{U}$的第$k$行,即求得$\mathbf{L}$与$\mathbf{U}$.
\[\begin{split}
u_{ki} = \frac{a_{ki} - \sum_{t = 1}^{k - 1}l_{kt}u_{ti}}{l_{kk}}, & \quad i = 1,2,\dots,n \\
l_{ik} = a_{ik} - \sum_{t = 1}^{k - 1}l_{kt}u_{ti}, & \quad i = 1,2,\dots,n 
\end{split}\]
\newline

\paragraph{16.} \textbf{证明} \\
\[\Vert\mathbf{A}^{-1}\Vert_{\infty} = \max_{\mathbf{x} \ne 0} \frac{\Vert \mathbf{A}^{-1}\mathbf{x}\Vert_{\infty}}{\Vert \mathbf{x} \Vert_{\infty}}\]
令$\mathbf{y} = \mathbf{A}^{-1}\mathbf{x}$,则有
\[\Vert\mathbf{A}^{-1}\Vert_{\infty} = \max_{\mathbf{y} \ne 0} \frac{\Vert \mathbf{y}\Vert_{\infty}}{\Vert \mathbf{Ay} \Vert_{\infty}}\]
即得
\[\frac{1}{\Vert\mathbf{A}^{-1}\Vert_{\infty}} = \min_{\mathbf{y} \ne 0} \frac{\Vert \mathbf{Ay}\Vert_{\infty}}{\Vert \mathbf{y} \Vert_{\infty}}\]
证毕.
\newline

\paragraph{20.} \textbf{证明} \\
由$\Vert \mathbf{AB} \Vert \le \Vert \mathbf{A} \Vert \Vert \mathbf{B} \Vert$ 得
\[\begin{split}
\mathrm{cond}(\mathbf{AB}) & = \Vert (\mathbf{AB})^{-1} \Vert \Vert \mathbf{AB} \Vert \\
& \le \Vert \mathbf{A}^{-1} \Vert \Vert \mathbf{B}^{-1} \Vert \Vert \mathbf{A} \Vert \Vert \mathbf{B} \Vert \\
& = \mathrm{cond}(\mathbf{A})\mathrm{cond}(\mathbf{B})
\end{split}\]
证毕.

\end{document}