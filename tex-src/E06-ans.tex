\documentclass[12pt, a4paper]{article}
\usepackage{ctex}

\usepackage{float}
\usepackage[super, square]{natbib}
\usepackage[margin = 1in]{geometry}
\usepackage{
  color,
  clrscode,
  amssymb,
  ntheorem,
  amsmath,
  listings,
  fontspec,
  xcolor,
  supertabular,
  multirow
}
\definecolor{bgGray}{RGB}{36, 36, 36}
\usepackage[
  colorlinks,
  linkcolor=bgGray,
  anchorcolor=blue,
  citecolor=black
]{hyperref}
\newfontfamily\courier{Courier}

\theoremstyle{margin}
\theorembodyfont{\normalfont}

\newtheorem{theorem}{Theorem}
\newtheorem{definition}[theorem]{Definition}
\newtheorem{example}[theorem]{Example}

\newcommand{\st}{\text{s.t.}}
\newcommand{\mn}{\mathnormal}
\newcommand{\tbf}{\textbf}
\newcommand{\fl}{\mathnormal{fl}}
\newcommand{\f}{\mathnormal{f}}
\newcommand{\g}{\mathnormal{g}}
\newcommand{\R}{\mathbf{R}}
\newcommand{\Q}{\mathbf{Q}}
\newcommand{\JD}{\textbf{D}}
\newcommand{\rd}{\mathrm{d}}
\newcommand{\str}{^*}
\newcommand{\vep}{\varepsilon}
\newcommand{\lhs}{\text{L.H.S}}
\newcommand{\rhs}{\text{R.H.S}}
\newcommand{\con}{\text{Const}}
\newcommand{\oneton}{1,\,2,\,\dots,\,n}
\newcommand{\aoneton}{a_1a_2\dots a_n}
\newcommand{\xoneton}{x_1,\,x_2,\,\dots,\,x_n}


\title{科学计算 Exercise 6}
\author{范舟\\516030910574\\致远学院2016级ACM班}
\date{}

\begin{document}

\lstset{numbers=left,
  basicstyle=\scriptsize\courier,
  numberstyle=\tiny\courier\color{red!89!green!36!blue!36},
  language=C++,
  breaklines=true,
  keywordstyle=\color{blue!70},commentstyle=\color{red!50!green!50!blue!50},
  morekeywords={},
  stringstyle=\color{purple},
  frame=shadowbox,
  rulesepcolor=\color{red!20!green!20!blue!20}
}
\maketitle

1. \paragraph{解} (1) 分别代入$f(x) = 1, f(x) = x, f(x) = x^2$,得方程组
\[\begin{cases}
A_{-1} + A_{0} + A_{1} = 2h \\
-hA_{-1} + hA_{1} = 0 \\
h^2A_{-1} + h^2A_{1} = \frac{2h^3}{3}
\end{cases}\]
解得
\[\begin{cases}
A_{-1} = \frac{h}{3} \\
A_{0} = \frac{4h}{3}\\
A_{1} = \frac{h}{3}
\end{cases}\]
因此此求积公式至少有2次代数精度,分别代入$f(x) = x^3, f(x) = x^4$,检验得
\[\begin{split}
& \int_{-h}^{h} x^3 \rd x = 0 = A_{-1}(-h)^3 + A_{0}0^3 + A_{1}h^3 \\
& \int_{-h}^{h} x^4 \rd x \ne A_{-1}(-h)^4 + A_{0}0^4 + A_{1}h^4
\end{split}\]
因此此求积公式有3次代数精度. 

(2) 分别代入$f(x) = 1, f(x) = x, f(x) = x^2$,得方程组
\[\begin{cases}
A_{-1} + A_{0} + A_{1} = 4h \\
-hA_{-1} + hA_{1} = 0 \\
h^2A_{-1} + h^2A_{1} = \frac{16h^3}{3}
\end{cases}\]
解得
\[\begin{cases}
A_{-1} = \frac{8h}{3} \\
A_{0} = \frac{-4h}{3}\\
A_{1} = \frac{8h}{3}
\end{cases}\]
因此此求积公式至少有2次代数精度,分别代入$f(x) = x^3, f(x) = x^4$,检验得
\[\begin{split}
& \int_{-2h}^{2h} x^3 \rd x = 0 = A_{-1}(-h)^3 + A_{0}0^3 + A_{1}h^3 \\
& \int_{-2h}^{2h} x^4 \rd x \ne A_{-1}(-h)^4 + A_{0}0^4 + A_{1}h^4
\end{split}\]
因此此求积公式有3次代数精度. 

(3) 代入$f(x) = 1$,检验知此积分公式至少有0阶代数精度.分别代入$f(x) = x, f(x) = x^2$,得方程组
\[\begin{cases}
(-1 + 2x_1 + 3x_2) / 3 = 0 \\
(-1 + 2x_1^2 + 3x_2^2) / 3 = \frac{2}{3}
\end{cases}\]
解得
\[\begin{cases}
x_1 = \frac{1 \pm \sqrt{6}}{5} \\
x_2 = \frac{3 \mp 2\sqrt{6}}{15}
\end{cases}\]
因此此求积公式至少有2次代数精度,代入$f(x) = x^3$,检验得
\[\int_{-1}^{1} x^3 \rd x \ne [(-1)^3 + 2x_1^3 + 3x_2^3] / 3\]
因此此求积公式有2次代数精度. 

(4) 分别代入$f(x) = 1, f(x) = x$,检验知此积分公式至少有1阶代数精度.代入$f(x) = x^2$,得
\[\frac{h^3}{2} - 2ah^3 = \frac{h^3}{3}\]
得
\[a = \frac{1}{12}\]
因此此求积公式至少有2次代数精度,分别代入$f(x) = x^3, f(x) = x^4$,检验得
\[\begin{split}
& \int_{0}^{h} x^3 \rd x = \frac{h^4}{4} = h(0 + h^3) / 2 + ah^2(0 - 3h^2)\\
& \int_{0}^{h} x^4 \rd x \ne h(0 + h^4) / 2 + ah^2(0 - 4h^3)
\end{split}\]
因此此求积公式有3次代数精度. 
\newline

6. \paragraph{解} 复合梯形公式的余项为
\[R_n(f) = -\frac{b - a}{12}h^2f''(\eta)\]
则有
\[\frac{e}{12}\frac{1}{n^2} \le \frac{1}{2} \times 10^{-5} \quad \Rightarrow \quad n \ge 213\]
即最小分为213等份.
复合辛普森求积公式的余项为
\[R_n(f) = -\frac{b - a}{180}(\frac{h}{2})^4f^{(4)}(\eta)\]
则有
\[\frac{e}{180}(\frac{1}{2n})^4 \le \frac{1}{2} \times 10^{-5} \quad \Rightarrow \quad n \ge 4\]
即最小分为4等份.
\newline

10. \paragraph{解} 在权函数为$\rho(x) = \frac{1}{\sqrt{x}}$时,首项系数为1的正交多项式前三项为
\[\varphi_0(x) = 1, \quad \varphi_1(x) = x - \frac{1}{3}, \quad \varphi_2(x) = x^2 - \frac{6}{7}x + \frac{3}{35}\]
取$x_0, x_1$为$\varphi_2(x)$的两个零点,则
\[x_0 = \frac{15 - 2\sqrt{30}}{35}, \quad x_1 = \frac{15 + 2\sqrt{30}}{35}\]
分别代入$f(x) = 1, f(x) = x$,得方程组
\[\begin{cases}
A_0 + A_1 = 2 \\
A_0 x_0 + A_1 x_1 = \frac{2}{3}
\end{cases}\]
解得
\[\begin{cases}
A_0 = 1 + \frac{\sqrt{30}}{18} \\
A_1 = 1 - \frac{\sqrt{30}}{18}
\end{cases}\]
因此,高斯型求积公式为
\[\int_{0}^{1}\frac{1}{\sqrt{x}}f(x) \rd x \approx (1 + \frac{\sqrt{30}}{18})f(\frac{15 - 2\sqrt{30}}{35}) + (1 - \frac{\sqrt{30}}{18})f(\frac{15 + 2\sqrt{30}}{35})\]

\end{document}