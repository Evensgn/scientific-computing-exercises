\documentclass[12pt, a4paper]{article}
\usepackage{ctex}

\usepackage{float}
\usepackage[super, square]{natbib}
\usepackage[margin = 1in]{geometry}
\usepackage{
  color,
  clrscode,
  amssymb,
  ntheorem,
  amsmath,
  listings,
  fontspec,
  xcolor,
  supertabular,
  multirow
}
\definecolor{bgGray}{RGB}{36, 36, 36}
\usepackage[
  colorlinks,
  linkcolor=bgGray,
  anchorcolor=blue,
  citecolor=black
]{hyperref}
\newfontfamily\courier{Courier}

\theoremstyle{margin}
\theorembodyfont{\normalfont}

\newtheorem{theorem}{Theorem}
\newtheorem{definition}[theorem]{Definition}
\newtheorem{example}[theorem]{Example}

\newcommand{\st}{\text{s.t.}}
\newcommand{\mn}{\mathnormal}
\newcommand{\tbf}{\textbf}
\newcommand{\fl}{\mathnormal{fl}}
\newcommand{\f}{\mathnormal{f}}
\newcommand{\g}{\mathnormal{g}}
\newcommand{\R}{\mathbf{R}}
\newcommand{\Q}{\mathbf{Q}}
\newcommand{\JD}{\textbf{D}}
\newcommand{\rd}{\mathrm{d}}
\newcommand{\str}{^*}
\newcommand{\vep}{\varepsilon}
\newcommand{\lhs}{\text{L.H.S}}
\newcommand{\rhs}{\text{R.H.S}}
\newcommand{\con}{\text{Const}}
\newcommand{\oneton}{1,\,2,\,\dots,\,n}
\newcommand{\aoneton}{a_1a_2\dots a_n}
\newcommand{\xoneton}{x_1,\,x_2,\,\dots,\,x_n}


\title{科学计算 Exercise 07a}
\author{范舟\\516030910574\\致远学院2016级ACM班}
\date{}

\begin{document}
\lstset{numbers=left,
  basicstyle=\scriptsize\courier,
  numberstyle=\tiny\courier\color{red!89!green!36!blue!36},
  language=Matlab,
  breaklines=true,
  keywordstyle=\color{blue!70},commentstyle=\color{red!50!green!50!blue!50},
  morekeywords={},
  stringstyle=\color{purple},
  frame=shadowbox,
  rulesepcolor=\color{red!20!green!20!blue!20}
}
\maketitle

\paragraph{2. 解}

计算知$f(x) = x^3 - x^2 - 1$在$[1.4, 1.5]$连续且有$f(1.4) f(1.5) < 0$,因此待求的根在$[1.4, 1.5]$中,将其记作$x_*$.

\paragraph{(1)} 
\[\varphi(x) = 1 + \frac{1}{x^2}, \quad \varphi'(x) = \frac{-2}{x^3}\]
$\varphi'(x)$在不动点$x_*$附近连续且单调,有
\[|\varphi'(x_*)| < |\varphi'(1.4)| = 0.7289 < 1\]
因此迭代公式$x = 1 + 1 / x^2$局部收敛.又有$\varphi'(x_*) \ne 0$,因此此迭代公式在$x_*$附近是线性收敛的.

\paragraph{(2)} 
\[\varphi(x) = \sqrt[3]{1 + x^2}, \quad \varphi'(x) = \frac{2x}{3} (1 + x^2)^{-\frac{2}{3}}\]
$\varphi'(x)$在不动点$x_*$附近连续,且有
\[|\varphi'(x)| < 0.5, \quad \forall x \in [1.4, 1.5]\]
因此$|\varphi'(x_*)| < 0.5 < 1$,迭代公式$x = 1 + 1 / x^2$局部收敛.又有$\varphi'(x_*) \ne 0$,因此此迭代公式在$x_*$附近是线性收敛的.

\paragraph{(3)} 
\[\varphi(x) = \frac{1}{\sqrt{x - 1}}, \quad \varphi'(x) = -\frac{1}{2} (x - 1)^{-\frac{3}{2}}\]
$\varphi'(x)$在不动点$x_*$附近连续且单调,有
\[|\varphi'(x)| \ge |\varphi'(1.5)| = 1.4142 > 1, \quad \forall x \in [1.4, 1.5]\]
下面证明此迭代公式在$x_*$附近不收敛.使用反证法,假设存在一收敛于$x_*$的迭代数列${x_k}$,则对于任一$\varepsilon_0$(满足$(x_* - \varepsilon_0, x_* + \varepsilon_0) \subset [1.4, 1.5]$),存在$N$满足$\forall n >= N, \quad |x_n - x_*| < \varepsilon_0$.又有
\[|x_{N + k} - x_*| = |\varphi'(\eta)||x_{N + k - 1} - x_*| > 1.4 |x_{N + k - 1} - x_*| > 1.4^k |x_N - x_*|\]
因此存在足够大的$k$,使得$|x_{N + k} - x_*| > \varepsilon_0$,与迭代数列收敛的条件矛盾,因此此迭代公式在$x_*$附近不收敛.

\paragraph{计算} 选用迭代公式(2),已知
\[|\varphi'(x)| < L = 0.5, \quad \forall x \in [1.4, 1.5]\]
根据误差关系
\[|x_* - x_k| \le \frac{1}{1 - L} |x_{k + 1} - x_k|\]
可在满足如下条件时停止迭代
\[\frac{1}{1 - L} |x_{k + 1} - x_k| = 2|x_{k + 1} - x_k| \le \frac{1}{2} \times 10^{-3} \quad \Rightarrow \quad |x_{k + 1} - x_k| \le \frac{1}{4} \times 10^{-3}\]
使用下面的 Matlab 代码计算
\begin{lstlisting}
prev_x = 1.5;
while true
    x = (1 + prev_x^2)^(1 / 3);
    if abs(x - prev_x) <= 1 / 4 * 10^(-3)
        break;
    end
    prev_x = x;
end
fprintf("%.3f\n", x);
\end{lstlisting}
每次迭代后的结果如下,取最终结果保留4位有效数字得近似根为$1.466$.
\begin{lstlisting}
1.5000
1.4812
1.4727
1.4688
1.4670
1.4662
1.4659
1.4657
\end{lstlisting}

\paragraph{5. 解}

根据 Steffensen 迭代公式
\[\psi(x) = x - \frac{[\varphi(x) - x]^2}{\varphi(\varphi(x)) - 2\varphi(x) + x}\]
用 Matlab 编写 Steffensen 迭代法的通用程序如下,其中phi为原始迭代函数的函数句柄,eps为所要求的精度.
\begin{lstlisting}
function ret = Steffensen(x0, phi, eps)
    prev_x = x0;
    while true
        x = prev_x - (phi(prev_x) - prev_x)^2 / ...
            (phi(phi(prev_x)) - 2 * phi(prev_x) + prev_x);
        if abs(x - prev_x) < 0.1 * eps
            break;
        end
        prev_x = x;
    end
    ret = x;
end
\end{lstlisting}
对第2题中(2),(3)迭代公式分别调用此函数,
\begin{lstlisting}
x0 = 1.5;
eps = 10^(-4);
ans2 = Steffensen(x0, @phi2, eps);
ans3 = Steffensen(x0, @phi3, eps);
fprintf("%.5f %.5f\n", ans2, ans3);

function y = phi2(x)
    y = (1 + x^2)^(1 / 3);
end

function y = phi3(x)
    y = 1 / sqrt(x - 1);
end
\end{lstlisting}
迭代公式(2)每次迭代后的结果如下
\begin{lstlisting}
1.500000
1.465558
1.465571
1.465571
\end{lstlisting}
迭代公式(3)每次迭代后的结果如下
\begin{lstlisting}
1.500000
1.467342
1.465576
1.465571
\end{lstlisting}
二者均得到计算结果$1.46557$.

\paragraph{6. 解}

由于迭代函数$\varphi(x)$至少为三阶收敛,设$f(x_*) = 0$,则有
\[\begin{cases}
f(x_*) = 0 \\
f'(x_*) = 0 \\
f''(x_*) = 0
\end{cases}\]
首先有
\[\varphi' = 1 - p'f - pf' - q'f^2 - 2qff' = 0\]
代入$x = x_*$得
\[1 - p(x_*)f'(x_*) = 0 \quad \Rightarrow \quad p(x_*) = \frac{1}{f'(x_*)}\]
取$p(x) = 1 / f'(x)$,则有
\[\begin{split}
\varphi & = x - \frac{f}{f'} - qf^2 \\
\varphi' & = 1 - \frac{(f')^2 - ff''}{(f')^2} - q'f^2 - 2qff' \\
& = \frac{ff''}{(f')^2} - q'f^2 - 2qff'\\
\varphi''(x_*) &= \frac{f''(x_*)}{f'(x_*)} - 2[f'(x_*)]^2q(x_*) = 0 \\
& \Rightarrow \quad q(x_*) = \frac{f''(x_*)}{2[f'(x_*)]^3}
\end{split}\]
因此,可以取
\[\begin{cases}
p(x) = \frac{1}{f'(x)} \\
q(x) = \frac{f''(x)}{2[f'(x)]^3}
\end{cases}\]
即可满足题目要求,使迭代函数$\varphi(x)$至少三阶收敛.

\paragraph{10. 证明}

记$\varphi(x) = x - f(x) / f'(x)$,则$x_{k + 1} = \varphi(x_k)$
\[\varphi''(x_*) = \frac{f''(x_*)}{f'(x_*)}\]
因为牛顿迭代法二阶收敛,当$k \to \infty$时,有
\[P_k = \frac{x_{k} - x_*}{(x_{k - 1} - x_*)^2} \to \frac{\varphi''(x_*)}{2!} = \frac{f''(x_*)}{2f'(x_*)}\]
\[\frac{P_{k - 1}}{R_k} = \frac{(x_{k - 1} - x_*)}{(x_{k - 2} - x_*)^2}\frac{(x_{k - 1} - x_{k - 2})^2}{(x_k - x_{k - 1})} = \frac{(x_{k - 1} - x_{k - 2})^2}{(x_{k - 2} - x_*)^2}\frac{(x_{k - 1} - x_*)}{(x_k - x_{k - 1})}\]
记
\[Q_k = \frac{(x_k - x_{k - 1})}{(x_{k - 1} - x_*)}\]
当$k \to \infty$时,因牛顿迭代法二阶收敛,有
\[Q_k = \frac{-(x_{k - 1} - x_*) + (x_k - x_*)}{(x_{k - 1} - x_*)} = -1 + 0 = -1\]
\[\lim_{k \to \infty} \frac{P_{k - 1}}{R_k} = \lim_{k \to \infty} \frac{Q_{k - 1}^2}{Q_{k}} = -1\]
因此
\[\lim_{k \to \infty} R_{k} = -\lim_{k \to \infty} P_{k} = -\frac{f''(x_*)}{2f'(x_*)}\]

\paragraph{14. 解}

由题目条件知$x_* = \sqrt[n]{a}$,由10题可知
\[\lim_{k \to \infty} \frac{\sqrt[n]{a} - x_{k + 1}}{(\sqrt[n]{a} - x_k)^2} = -\lim_{k \to \infty} \frac{x_{k} - x_*}{(x_{k - 1} - x_*)^2} = -\frac{f''(x_*)}{2f'(x_*)}\]

\paragraph{方程1}
\[\begin{cases}
f(x) = x^n - a \\
f'(x) = nx^{n - 1} \\
f''(x) = n(n - 1)x^{n - 2}
\end{cases}\]
迭代公式为
\[x_{k + 1} = x_k - \frac{f(x_k)}{f'(x_k)} = x_k - \frac{x_k^n - a}{nx_k^{n - 1}}\]
\[\lim_{k \to \infty} \frac{\sqrt[n]{a} - x_{k + 1}}{(\sqrt[n]{a} - x_k)^2} = -\frac{f''(x_*)}{2f'(x_*)} = -\frac{n(n - 1)a^{\frac{n - 2}{n}}}{2na^{\frac{n - 1}{n}}} = \frac{1 - n}{2\sqrt[n]{a}}\]

\paragraph{方程2}
\[\begin{cases}
f(x) = 1 - ax^{-n} \\
f'(x) = anx^{-(n + 1)} \\
f''(x) = -an(n + 1)x^{-(n + 2)}
\end{cases}\]
迭代公式为
\[x_{k + 1} = x_k - \frac{f(x_k)}{f'(x_k)} = x_k - \frac{1 - ax_k^{-n}}{anx_k^{-(n + 1)}} = x_k(1 + \frac{1}{n}) - \frac{x_k^{n + 1}}{an}\]
\[\lim_{k \to \infty} \frac{\sqrt[n]{a} - x_{k + 1}}{(\sqrt[n]{a} - x_k)^2} = -\frac{f''(x_*)}{2f'(x_*)} = \frac{an(n + 1)a^{-\frac{(n + 2)}{n}}}{2ana^{-\frac{(n + 1)}{n}}} = \frac{n + 1}{2\sqrt[n]{a}}\]

\end{document}